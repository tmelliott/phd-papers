\documentclass[draft,a4paper,onecolumn]{IEEEtran}\usepackage[]{graphicx}\usepackage[]{color}
%% maxwidth is the original width if it is less than linewidth
%% otherwise use linewidth (to make sure the graphics do not exceed the margin)
\makeatletter
\def\maxwidth{ %
  \ifdim\Gin@nat@width>\linewidth
    \linewidth
  \else
    \Gin@nat@width
  \fi
}
\makeatother

\definecolor{fgcolor}{rgb}{0.345, 0.345, 0.345}
\newcommand{\hlnum}[1]{\textcolor[rgb]{0.686,0.059,0.569}{#1}}%
\newcommand{\hlstr}[1]{\textcolor[rgb]{0.192,0.494,0.8}{#1}}%
\newcommand{\hlcom}[1]{\textcolor[rgb]{0.678,0.584,0.686}{\textit{#1}}}%
\newcommand{\hlopt}[1]{\textcolor[rgb]{0,0,0}{#1}}%
\newcommand{\hlstd}[1]{\textcolor[rgb]{0.345,0.345,0.345}{#1}}%
\newcommand{\hlkwa}[1]{\textcolor[rgb]{0.161,0.373,0.58}{\textbf{#1}}}%
\newcommand{\hlkwb}[1]{\textcolor[rgb]{0.69,0.353,0.396}{#1}}%
\newcommand{\hlkwc}[1]{\textcolor[rgb]{0.333,0.667,0.333}{#1}}%
\newcommand{\hlkwd}[1]{\textcolor[rgb]{0.737,0.353,0.396}{\textbf{#1}}}%
\let\hlipl\hlkwb

\usepackage{framed}
\makeatletter
\newenvironment{kframe}{%
 \def\at@end@of@kframe{}%
 \ifinner\ifhmode%
  \def\at@end@of@kframe{\end{minipage}}%
  \begin{minipage}{\columnwidth}%
 \fi\fi%
 \def\FrameCommand##1{\hskip\@totalleftmargin \hskip-\fboxsep
 \colorbox{shadecolor}{##1}\hskip-\fboxsep
     % There is no \\@totalrightmargin, so:
     \hskip-\linewidth \hskip-\@totalleftmargin \hskip\columnwidth}%
 \MakeFramed {\advance\hsize-\width
   \@totalleftmargin\z@ \linewidth\hsize
   \@setminipage}}%
 {\par\unskip\endMakeFramed%
 \at@end@of@kframe}
\makeatother

\definecolor{shadecolor}{rgb}{.97, .97, .97}
\definecolor{messagecolor}{rgb}{0, 0, 0}
\definecolor{warningcolor}{rgb}{1, 0, 1}
\definecolor{errorcolor}{rgb}{1, 0, 0}
\newenvironment{knitrout}{}{} % an empty environment to be redefined in TeX

\usepackage{alltt}

\usepackage{amsmath}
\usepackage{amsfonts}
\usepackage{bm}

\title{Modelling a Public Transport System in Real Time using a Particle Filter}
\author{Tom~Elliott}
\IfFileExists{upquote.sty}{\usepackage{upquote}}{}
\begin{document}

\maketitle


\begin{abstract}
  Model all vehicles in the public transport system in real time.
  Particle filter model for each vehicle (bus),
  using observations of position (GPS coordinates) to infer distance into trip
  and velocity.
\end{abstract}

\begin{IEEEkeywords}
  Particle filter, transit, real-time
\end{IEEEkeywords}


\section{Introduction}
\label{sec:intro}


Basically talk about how the Kalman filter is good, fast, but not as reliable as we want.
Some faults, namely assumptions of normality, symmetry, and that we have to transform observations
(rather than transform parameters to match observations).
Particle filter has been used, and allow us to get around these problems.
Some examples of particle filter usage \ldots




\section{Particle Filtering}
\label{sec:pf}


Theory etc of how the particle filter works.




\section{GTFS Data and Road Segmentation}
\label{sec:gtfs}

Describe the data (GTFS static and real-time). (Maybe this go in section~\ref{sec:intro}?).

And the idea behind \emph{segmenting} bus routes between intersections.



\section{Transit Vehicle Transition Function}
\label{sec:transition}

The logic behind the transition function specific for a transit vehicle
(in particular, a bus).

Define all parameters, priors, etc.



\section{Likelihood Function}
\label{sec:likelihood}

The somewhat complex likelihood function.
Basic - distance from observation, \emph{or} proximity to stop/intersection.

More complex - include stop time updates.



\section{Implementation and Results}
\label{sec:results}

Implementation in \texttt{C++} and running in real-time.
Some pretty pictures of the results of the particle filter? 
How well it performs, etc.



\section{Discusion and Future Work}
\label{sec:discussion}

Talk about the good and the bad, and how we hope to use the particle filter in future to make predictions.



\end{document}
