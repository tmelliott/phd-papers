\section{Discussion and future work}
\label{sec:discussion}

In this paper, we described an approach to transit vehicle modelling
that enables the \rt estimation of road state from vehicle position data,
which will later be used to make arrival time predictions
that account for \rt traffic conditions.
The particle filter allows the modelling of complex bus behaviours,
providing estimates of vehicle travel times along roads
which are in turn used to estimate the road network state.


We have focussed on making a general framework
into which new models can be incorporated,
such as implementing intersection-specific behaviour
once intersection locations are available,
or exiting models changed.
For example, we plan to model stop dwell time using historical data,
which will allow us to replace the exponential distribution
with something more appropriate.


The network model presented here is simple,
but provides the foundation for future work to
develop a formal model combining \rt travel time information with historical data,
allowing short- and medium-term forecasts of road state,
for example incorporating peak time congestion.
As for the arrival times,
we plan to estimate these from a combination of historical
and \rt data.
The model presented in this paper provides the starting point
for a transit modelling and predictive framework
that can be adapted to specific transit providers,
combined with historical data,
and run in \rt to provide commuters with
improved ETAs with which they can more reliably
plan their commutes.


The results of our simulations show that the framework is a
feasible \rt alternative.
While the timings presented were for off-peak times,
further investigation into the choice of $N$ and $N_\text{eff}$,
as well as work removing redundancy from the transition algorithm,
should allow us to model all vehicles during peak periods---%
when there are 2--3~times as many vehicles---%
within our 30-second time frame target.
