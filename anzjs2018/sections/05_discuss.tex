\section{Discussion and Future Work}
\label{sec:discussion}

In this paper we described an approach to transit vehicle modelling
that enables the \rt estimation of road state from vehicle position data,
which in the future will be used to make arrival time predictions
that account for \rt traffic conditions.
The vehicle model, implemented using a particle filter,
allows complex bus behaviours to be modelled,
with future work planned to extend the model to handle physical intersections,
such as traffic lights.
From the particle filter,
vehicle travel times along roads are estimated which are in turn used 
in the modelling of the road network state.


The network model presented here is simple,
but provides the foundation for future work to
develop a formal model combining \rt travel time information with historical data, 
allowing short- and medium-term forecasts of road state,
for example incorporating peak time congestion.
This in turn will provide the basis for an arrival time prediction model
that combines \rt vehicle state with travel time forecasts
and provide commuters with what will hopefully be reliable ETAs.
These will likely consist of a point estimate along with a prediction interval,
as it is impossible given the available data to accurately predict arrival time
due to traffic lights, unknown dwell times at intermediate bus stops,
and many other factors.




% The main goal of this work was to explore the feasibility
% of a particle filter as a \rt estimation method for transit vehicle states,
% and the use of estimated states to determine traffic conditions
% throughout the network.
% This allows vehicles to inform others---irrespective of route---of 
% current \rt traffic conditions,
% and provide a means of obtaining more reliable ETAs.


% We have shown that the particle filter a viable \rt option in terms of speed,
% and while it occasionally loses the bus,
% a lot of this is attributed to invalid data (incorrect route, direction, etc).
% Future development will incorporate more accurate stopping behaviour,
% perhaps using GTFS arrival and departure times in the likelihood function.


% The network model is still in the early stages of development,
% but given that the implementation is running well within our 30~second target,
% there is room for developing a more complex model.
% Future work will explore the use of historical data to develop informative priors
% for travel times,
% which will provide a useful fallback in the case of no data,
% or for making more accurate long-term predictions.
% Once such a model has been developed, we plan to explore prediction estimates,
% particularly comparing point and intervals as a means of communicating 
% uncertainty to commuters.
