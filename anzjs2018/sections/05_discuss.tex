\section{Discussion and Future Work}
\label{sec:discussion}

In this paper we described an approach to transit vehicle modelling
that enables the \rt estimation of road state from vehicle position data,
which will later be used to make arrival time predictions
that account for \rt traffic conditions.
The vehicle model, implemented using a particle filter,
allows complex bus behaviours to be modelled,
from which 
vehicle travel times along roads are estimated which are in turn used 
in the modelling of the road network state.


We have focussed on making a general framework
in which the existing models used for specific components
can easily be changed---%
for example, using a more complex model of stop dwell times
based on historical data in place of the exponential---%
or new ones added, such as implementing intersection-specific behaviour
if intersecion locations are available.
Future work is planned to detect route meeting points
and use these as intersection nodes,
with the goal of further reducing road segment overlap.


The network model presented here is simple, 
but provides the foundation for future work to
develop a formal model combining \rt travel time information with historical data, 
allowing short- and medium-term forecasts of road state,
for example incorporating peak time congestion.
% This in turn will provide the basis for an arrival time prediction model
% that combines \rt vehicle state with travel time forecasts
% in order to provide commuters with more reliable ETAs.
As for the arrival times, 
we plan to estimate these from a combination of historical 
and \rt data.
The model presented in this paper provides the starting point
for a transit modelling and predictive framework
that can be adapted to specific transit providers,
combined with historical data,
and run in \rt to provide commuters with
improved ETAs with which they can more reliably
plan their commutes.


The results of our simulations show that the framework is a 
faesible \rt alternative.
While the timings presented were for off-peak times,
we hope futher investigation of choosing $N$ and $N_\text{eff}$,
as well as work removing some redudancy from the transition algorithm,
will mean times during peak periods, when there are 2-3~times
as many vehicles, will come in under our 30~second target.
