\section{Introduction}
\label{sec:intro}


In public transport, \rt information (RTI)---%
most notably estimated times of arrival (ETAs)---%
keeps transport users informed of changes to their commute,
allowing them to plan their journey accordingly.
Previous research has shown that perceived waiting time is less
when arrival time information is available \citep{TCRP_2003b},
but here in Auckland, RTI is highly unreliable,
leading to frustration and ultimately detering public transport users.
Inaccurate ETAs are a major source of this frustration,
as they fluctuate---sometimes dramatically---over time.
This can be caused by complex changes to traffic conditions,
but equally can be due simply to poor schedule calibration.
Furthermore, buses are shown as \emph{on-time} 
if they are not present in the \rt system and have not been manually cancelled,
in which case ETAs are based solely on scheduled arrival times.
Once the scheduled arrival time has passed,
the service is removed from the \rt board,
leaving passengers unsure as to when---if at all---their bus will show up,
a phenomenon referred to as ``ghost buses'' by transport bloggers.


Arrival time prediction is only as reliable as the underlying model,
and while a lot of work has gone into developing public transport vehicle models
\citep{Cathey_2003,Jeong_2005,Yu_2011,Hans_2015},
many public transport providers---notably Auckland Transport---useno formal model is used.
Instead, ETAs are based solely on the scheduled arrival time
adjusted by the vehicle's arrival or departure delay at the most recent stop, 
\emph{if available},
the primary assumption being that the schedule is valid,
and secondly that there is no unusual congestion along the route;
neither of these are valid,
particularly in our test area of Auckland, New Zealand,
where infrastructure for buses (such as priority lanes) is limited,
and bus drivers, in general, 
do not actively adhere to the schedule.


A more robust modelling and prediction framework 
based on \rt congestion information would avoid making these assumptions.
Such a framework should consist of a robust vehicle model to estimate the position and speed
of transit vehicles from \rt GPS data,
and secondly a means of combining speed information from vehicles
to model traffic flows,
which can then be used to improve arrival time predictions.
Several vehicle modelling approaches were explored, 
including the Kalman filter \citep{Dailey_2001,Cathey_2003},
machine learning models \citep{Yu_2006,Chang_2010},
and the particle filter \citep{Hans_2015}.


The particle filter has proven itself as a robust option for
\rt vehicle tracking applications
\citep{Gustafsson_2002,Davidson_2011},
and as computational power has increased over recent years,
a strong competitor for \rt applications.
\cite{Ulmke_2006} showed it to handle situations where traditional
(i.e., Kalman filter) methods break down:
the particle filter retained the accurate uncertainty about the location
where there was a lack of information,
an important property for our model in which we expect
many situations where there are sparse observations
and a strongly multimodal distribution.
By using a particle filter,
we were able to develop a simple, flexible framework
which is robust to the type of data we expect to observe.


The second component involves estimating traffic conditions throughout the transport network
using the information obtained from the vehicle model.
\cite{Yu_2010} improved prediction accuracy by using the travel times
of buses travelling along the same route.
A similar method presented by \cite{Hans_2015}
used headway, the time between consecutive vehicles at a point on the route,
as a predictor of travel time.
Since these approaches only work well on high frequency routes,
\cite{Yu_2011} showed further improvements by combining travel times 
from several routes;
however, this was limited to predefined converging routes.
In general, however, no comprehensive network modelling approach has been proposed using
solely GPS position data to model and account for congestion when estimating arrival times.


In Section~\ref{sec:models} of this paper, 
we describe the \rt vehicle and network state models
used to estimate traffic congestion using the transit vehicles
travelling through the network in real-time,
and in Section~\ref{sec:rt} we demonstrate the feasibility of using a particle filter
in \rt to model transit vehicles,
and evaluate its performance.
First, however, we descibe how we construct a
transit road network from GTFS data in Section~\ref{sec:gtfs}.



