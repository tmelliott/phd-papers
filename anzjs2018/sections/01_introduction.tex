\section{Introduction}
\label{sec:intro}


From the comments posted on social media during outages,
it would seem that \rt information (RTI) has surpassed being a novelty
to become an integral component of public transport.
In Auckland, the RTI provided by Auckland Transport is unreliable at best,
and often succumbs to several major problems that public transport users
are all too familiar with.
The first is inaccurate estimated times of arrival (ETAs),
which tend to increase over time as the bus gets progressively later
due to traffic congestion, or simply because the schedule is poorly calibrated.
The second is buses that are shown as on time but in fact are not running,
and therefore have not registered on the \rt system in which case the default,
scheduled times are used as ETAs.


Arrival time prediction is only as accurate as the underlying model,
and while a lot of work has gone into developing public transport vehicle models
\citep{Cathey_2003,Jeong_2005,Yu_2011,Hans_2015},
in many public transport systems no formal model is used.
Instead, ETAs are solely based on the scheduled arrival time
adjusted by the vehicle's delay at the most recent stop, if available.
This assumes that the schedule is valid,
and that there is no unusual congestion along the route,
neither of which are valid assumptions,
particularly in our test area of Auckland, New Zealand,
where infrastructure for buses (such as priority lanes) is very limited.


In order to improve upon this, we first needed to develop a vehicle model
that could infer the position and speed of vehicles accurately.
Several options were explored, 
including the Kalman filter models used by \cite{Dailey_2001} and \cite{Cathey_2003},
machine learning models as proposed by \cite{Yu_2006} and \cite{Chang_2010},
however finally decided on using a particle filter approach as demonstrated
by \cite{Hans_2015}.
The main reason for this choice is that particle filter models can be 
flexible and aren't limited by many assumptions,
instead only by computational demands.


The second consideration is how to make \emph{reliable} arrival time estimates.
Ideally, congestion of intermediate roads should be incorporated into the predictive model,
and one way of doing this is to use the travel time of recent buses along
the same roads to improve ETA reliability.
In the past, several novel attempts at this were made,
from previous-trip (of the same route) as presented by \cite{Yu_2010},
and more complex versions of this where routes that congverged could
inform on each other \citep{Yu_2011}.
In general, however, no comprehensive network modeling approach has been attempted
as far as being trained and used for public transport predictions.


In this applications paper, we describe a real-time vehicle and network state
model which uses the vehicles traveling throughout the transit road network
to update road travel time information.
The vehicle model is implemented using a particle filter,
so we also demonstrate the faesibility of this as a valid real-time option.
From the particle filter estimates of vehicle trajectories we 
can estimate the vehicles travel time along roads within the network,
and use these to update the network state.
This can then be used in future work to develop a more reliable 
arrival time estimation method.







% The need for realtime information.
% Where it goes wrong: 
% - bad schedule calibration;
% - no accounting for realtime traffic 
% - mention how only the delays are used (in Akl), which rely on (often inaccurate) schedules
% Literature review of historical and present methods?

% % Bus arrival-time prediction is far from a novel concept: 
% % ever since the first vehicle tracking technology was implemented
% % in transit vehicles in the late 1980's, 
% % real-time models have been implemented to provide \gls{rti} for commuters.
% % The simplest form of this is, the \gls{eta}
% % (usually displayed as \emph{minutes until arrival}),
% % was implemented in {{region}} by \cite{Wall99analgorithm} 
% % using their proposed \gls{kf} algorithm.
% % In their work, they used historical data to obtain \glspl{eta}.

% % Over the years, advances in computing power have enabled new
% % modeling approaches,
% % such as support vector machines \citep{Yu_2006},
% % artificial neural networks \citep{Yu_2011},
% % and more recently particle filters \citep{Hans_2015}.



% Particle filter has proven useful to this kind of dynamic problem,
% where the state distribution is non-gaussian, non-symmetric, 
% multimodal...

% Any other recent attempts at network-estimation?
% There was a KF using point-to-point speed estimations? 

% Computational aspect - historically, models had to be simple 
% (e.g., KF) so they could run in realtime.
% These days, running a somewhat powerful VM on a remote server
% is almost standard practice, so we can take full advantage 
% and use a more computationally demanding (particle filter)
% but more accurate estimation technique - paired with a more complex model.

% We first give an overview of the transit network,
% and how we construct it from raw GTFS shape data.
% This paper assumes a very basic model,
% with a focus on the implementation in a realtime setting.
