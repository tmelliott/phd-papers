\section{Introduction}
\label{sec:intro}



In public transport, \rt information (RTI), for example estimated times of arrival (ETAs),
keeps transport users informed about their journey,
and in some places make decisions on the best route to take.
Unfortunately, here in Auckland, RTI is highly unreliable,
and succumbs to several problems that public transport users
are all too familiar with.
One of these is inaccurate ETAs,
which tend to either increase or decrease over time as the bus gets progressively later or earlier
due to traffic congestion, or  because the schedule is poorly calibrated.
A second is that buses are shown as on time but in fact are not running,
and therefore have not registered on the \rt system.
In this case, the scheduled times are used for ETAs and, 
when the bus doesn't show up, 
the bus disappeared from the \rt board and we get what passengers have described as ``Ghost Buses''.

Arrival time prediction is only as accurate as the underlying model,
and while a lot of work has gone into developing public transport vehicle models
\citep{Cathey_2003,Jeong_2005,Yu_2011,Hans_2015},
in many public transport systems no formal model is used.
Instead, ETAs are solely based on the scheduled arrival time
adjusted by the vehicle's delay at the most recent stop, if available.
This assumes that the schedule is valid,
and that there is no unusual congestion along the route,
neither of which are valid assumptions,
particularly in our test area of Auckland, New Zealand,
where infrastructure for buses (such as priority lanes) is limited.
This behaviour is exaggerated when there is a large distance between stops,
as the ETA can change drastically.


To avoid these problems and improve the reliability of ETAs,
we first needed to develop a model to infer position and speed
of transit vehicles from \rt GPS data,
isntead of relying on delays reported only at stops.
Several modeling approaches were explored, 
such as the Kalman filter (used by \cite{Dailey_2001} and \cite{Cathey_2003}, ...),
machine learning models \citep{Yu_2006,Chang_2010},
and a particle filter approach demonstrated
by \cite{Hans_2015}.
We chose to use the last of these, the particle filter,
as they are very flexible and aren't limited by assumptions
such as unimodality;
instead, they are constrained by computational complexity.


The \emph{reliablility} of arrival time estimates
alo needs to be considered.
Ideally, congestion along intermediate roads should be incorporated into the predictive model,
and one way of doing this is to use the travel time of preceeding buses along the same roads.
Several such approaches have been proposed.
\cite{Yu_2010} used the travel time of vehilces on the same \emph{routes},
which they showed to improve ETAs.
However, this only works on routes with a high frequency.
\citep{Yu_2011} extended this by looking at roads where several major rouets converged,
again showing improvements but limited in scope.
In general, however, no comprehensive network modeling approach has been proposed that used
solely GPS position data to estimate arrival times
while accounting for congestion.


In this paper, we describe a \rt vehicle and network state
model which uses the vehicles traveling throughout the transit road network
to update congestion information in \rt.
The vehicle model is implemented using a particle filter,
so it is also necessary to demonstrate the faesibility of this as a valid real-time option.
From the particle filter estimates of vehicle trajectories,  
we estimate the travel time along roads update the road network's state.
Future work will involve using this travel time information 
in combination with historical data to develop an arrival time prediction model.




