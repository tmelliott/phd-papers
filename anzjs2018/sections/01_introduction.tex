\section{Introduction}
\label{sec:intro}



In public transport, \rt information (RTI), for example estimated times of arrival (ETAs),
keeps transport users informed about their journey,
and in some places make decisions on the best route to take.
Unfortunately, here in Auckland, RTI is highly unreliable,
and succumbs to several problems that public transport users
are all too familiar with.
One of these is inaccurate ETAs,
which tend to either increase or decrease over time as the bus gets progressively later or earlier
due to traffic congestion, or  because the schedule is poorly calibrated.
Another is that buses are shown as \emph{on-time} if they are not present
in the \rt system,
no matter whether this is due to technical problems,
or that the bus has failed to show up;
in this case, schedule times are used to determine ETAs and, 
if the bus doesn't show up on time,
the service is removed from the \rt board 
and we get what passengers have described as ``Ghost Buses''.

Arrival time prediction is only as accurate as the underlying model,
and while a lot of work has gone into developing public transport vehicle models
\citep{Cathey_2003,Jeong_2005,Yu_2011,Hans_2015},
in many public transport systems no formal model is used.
Instead, ETAs are solely based on the scheduled arrival time
adjusted by the vehicle's delay at the most recent stop, if available.
This assumes that the schedule is valid,
and that there is no unusual congestion along the route,
neither of which are valid assumptions,
particularly in our test area of Auckland, New Zealand,
where infrastructure for buses (such as priority lanes) is limited.
This behaviour is exaggerated when there is a large distance between stops,
as the ETA is only updated at stops so there can be a long time between updates.


To avoid these problems and improve the reliability of ETAs,
we required two things:
first, a robust vehicle model to estimate the position and speed
of transit vehicles from \rt GPS data;
and secondly, a way to incorporate road congestion information 
into the arrival time predictions.
Several vehicle modeling approaches were explored, 
such as the Kalman filter (used by \cite{Dailey_2001} and \cite{Cathey_2003}, ...),
machine learning models \citep{Yu_2006,Chang_2010},
and a particle filter approach demonstrated
by \cite{Hans_2015}.
The last of these, the particle filter, was chosen for several reasons,
discussed in Section~\ref{sec:pf}.


In order to improve the reliability of ETAs,
congestion along intermediate roads should be incorporated into the predictive model.
One solution is to use the travel time of preceeding buses along the same roads
to predict travel times for the current bus,
such as proposed by \cite{Yu_2010},
who were able to improve prediction accuracy
by using the travel time of vehicles on the same \emph{route}.
However, this only works well for high-frequency routes, so
\cite{Yu_2011} modeled travel times along roads where several major routes converged,
which showed further improvement however is limited in scope.
In general, however, no comprehensive network modeling approach has been proposed us
solely GPS position data to estimate arrival times
while accounting for congestion.


In this paper, we describe a \rt vehicle and network state
model which uses the vehicles traveling throughout the transit road network
to update congestion information in \rt.
As the vehicle model is implemented using a particle filter,
we demonstrate its faesibility as a valid real-time option.
From the particle filter estimates of vehicle trajectories,  
The particle filter provides estimates of vehicle travel times along roads,
which can be used to update the road network's state.
Future work involves using this travel time information 
in combination with historical data to develop an arrival time prediction model.




