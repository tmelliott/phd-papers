\section{Introduction}
\label{sec:intro}


The need for realtime information.
Where it goes wrong: 
- bad schedule calibration;
- no accounting for realtime traffic 
- mention how only the delays are used (in Akl), which rely on (often inaccurate) schedules
Literature review of historical and present methods?

% Bus arrival-time prediction is far from a novel concept: 
% ever since the first vehicle tracking technology was implemented
% in transit vehicles in the late 1980's, 
% real-time models have been implemented to provide \gls{rti} for commuters.
% The simplest form of this is, the \gls{eta}
% (usually displayed as \emph{minutes until arrival}),
% was implemented in {{region}} by \cite{Wall99analgorithm} 
% using their proposed \gls{kf} algorithm.
% In their work, they used historical data to obtain \glspl{eta}.

% Over the years, advances in computing power have enabled new
% modeling approaches,
% such as support vector machines \citep{Yu_2006},
% artificial neural networks \citep{Yu_2011},
% and more recently particle filters \citep{Hans_2015}.



Particle filter has proven useful to this kind of dynamic problem,
where the state distribution is non-gaussian, non-symmetric, 
multimodal...

Any other recent attempts at network-estimation?
There was a KF using point-to-point speed estimations? 

Computational aspect - historically, models had to be simple 
(e.g., KF) so they could run in realtime.
These days, running a somewhat powerful VM on a remote server
is almost standard practice, so we can take full advantage 
and use a more computationally demanding (particle filter)
but more accurate estimation technique - paired with a more complex model.

We first give an overview of the transit network,
and how we construct it from raw GTFS shape data.
This paper assumes a very basic model,
with a focus on the implementation in a realtime setting.
