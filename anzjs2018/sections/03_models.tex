\section{The Models}
\label{sec:models}

\subsection{Real-time vehicle model}
\label{sec:pf}

In order to estimate vehicle speed from a series of noisy GPS observations,
we need a model capable of overcoming several key problems:
varying update intervals, multimodality, assymetry, 
and some other artefacts such as mentioned in Section~\ref{sec:realtime_data}.
One estimation technique that suits itself well to this type of problem
is the particle filter, 
which has been used in several other bus modeling applications
\citep{Hans_2015}.
The primary advantage of this approach is that we need not make any 
unnecessary assumptions, particularly on the shape of the distributions 
as is the case in other commonly used methods such as the Kalman filter.


To generate arrival time predictions, 
several aspects of the vehicle's \emph{state} need to first be estimated,
such as its speed.
The \emph{unmeasurable state} of a vehicle at time $t_k$ is denoted by
\begin{equation}
    X_k = \begin{bmatrix} x_k & \dot x_k & \ddot x_k \end{bmatrix}^\top,
\end{equation}
where $x$ is the distance traveled along a route,
$\dot x$ is the speed (the first derivative of distance traveled as a function of time),
and $\ddot x$ is acceleration (the second derivative).




\subsection{Network model}
\label{sec:kf}

Two parts to this - first, the prior prediction step; then, the data update step.
