\section{The Models}
\label{sec:models}

Real-time tracking applications often use
a Bayesian filtering approach,
or recursive Bayesian estimation,
in which the data is processed sequentially instead of all at once,
reducing computational demands
\citep{Anderson_2012}.
These models assume an underlying Markov process with state $\boldsymbol{\mathcal{X}}_k$,
an $n\times1$ column vector,
and system noise $\boldsymbol{\omega}_k\sim\mathrm{N}(\boldsymbol{0},\mathcal{Q})$ 
of which observations $\boldsymbol{\mathcal{Y}}_k$,
an $m\times1$ column vector, are made
with error $\boldsymbol{\nu}_k\sim\mathrm{N}(\boldsymbol{0},\mathcal{R})$,
giving the following general model:
\begin{equation}
\label{eq:rbe_model}
\begin{split}
\boldsymbol{\mathcal{X}}_k &= f(\boldsymbol{\mathcal{X}}_{k-1}, \boldsymbol{\omega}_k) \\
\boldsymbol{\mathcal{Y}}_k &= h(\boldsymbol{\mathcal{X}}_k) + \boldsymbol{\nu}_k
\end{split}
\end{equation}
The transition function $f:\mathbb{R}^n\mapsto\mathbb{R}^n$ 
describes the relationship between consecutive states,
while the measurement function $h:\mathbb{R}^n\mapsto\mathbb{R}^m$ is a deterministic function
mapping the underlying state to the observable state.



The following sections describe the two models used in this application,
both of which are examples of a Bayesian filter.
The first, implemented using a particle filter, estimates vehicle states
to infer travel times along roads from a sequence of GPS positions;
the second uses these travel times to update road states 
and is implemented using an \emph{information filter},
a variant of the Kalman filter \citep{Anderson_2012}.

\subsection{Real-time vehicle model}
\label{sec:pf}

The underlying vehicle state at time $t_k$ consists of
the vehicle's distance travelled $x$ in meters along the route and
its speed $\dot x$ in meters per second.
% The travel times along road segments, $\bz = (z_1,\ldots,z_L)^\top$, in seconds,
% are estimated sequentially as the vehicle traverses the route,
% and are not included explicitely in the state.
These are combined into the state vector
$\bX_k = \left[x_k\ \dot x_k\right]^\top$,
of which observations $\bY_k$ are made using a GPS at time $t_k$
with GPS error $\epsilon^2$,
giving the longitude $\lambda_k$ and latitude $\phi_k$ of the vehicle
as $\bY_k = \left[\lambda_k\ \phi_k \right]^\top$.

We use a particle filter to estimate $\bX_k$,
due to its high flexibility and use in recent 
transit vehicle modelling applications \citep{Hans_2015},
as well as providing an easy way to estimate travel times
$\bz = (z_1,\ldots,z_L)^\top$, in seconds, along road segments
as the vehicle traverses the route.
The primary advantage of a particle filter is its handling of multimodality,
as demonstrated in Figure~\ref{fig:pf_state_predict},
which is a common feature of the proposal distribution, particularly around bus stops.
Another advantage is that the likelihood function is intuitively based
on the physical distance between the vehicle observation and the state's
estimate of the true location, $h(\bX_k)$ (Section~\ref{sec:pf_update}),
rather than the reverse of first estimating the ``observed'' distance travelled.
Conversely, particle filter methods are computationally demanding,
requiring an increasing number of particles as the model complexity and
number of parameters increases \citep{Carpenter_1999}.
Section~\ref{sec:rt} describes our implementation of the particle filter in real-time 
and its timings.



\begin{figure}[p]
    \centering
    \begin{subfigure}[t]{0.9\textwidth}
        \centering
        \includegraphics[width=0.7\textwidth]{figures/03_particle_filter_1.pdf}
        \caption{
            Vehicle state is mapped to observation space by the
            measurement function $h$.   
        }
        % The vehicle state represented by a sample of discrete points is related to the observable state (GPS position) through the transition function $h$.}
        \label{fig:pf_state_prev}
    \end{subfigure}\\
    \begin{subfigure}[t]{0.9\textwidth}
        \centering
        \includegraphics[width=0.7\textwidth]{figures/03_particle_filter_2.pdf}
        \caption{
            The transition function $f$ predicts the future state 
            of each particle.
        }
            % The transition function $f$ models the behaviour of each particle,
            % including changes in acceleration and stopping behaviour at bus stops,
            % which can result in multiple modes.}
        \label{fig:pf_state_predict}
    \end{subfigure}\\
    % \begin{subfigure}[t]{0.48\textwidth}
    %     \centering
    %     \includegraphics[width=\textwidth]{figures/03_particle_filter_6.pdf}
    %     \caption{Each particle's distance from the true observation (red cross) 
    %         is calculated using
    %         the geographic distance between two coordinates, which can then be used
    %         to calculate it's weight.}
    %     \label{fig:pf_state_update}
    % \end{subfigure}\;\;
    \begin{subfigure}[t]{0.9\textwidth}
        \centering
        \includegraphics[width=0.7\textwidth]{figures/03_particle_filter_4.pdf}
        \caption{
            The state is updated by weighted resampling based on distance
            of particle to observation.
        }
        % Resampling occurs using likelihood weights that use the distance
        %     between the particle's location and the vehicles reported location (red cross),
        %     resulting in the updated state.}
        \label{fig:pf_state_predict2}
    \end{subfigure}
    \caption{
        The particle filter approximates vehicle state using a set of 
        discrete points (a), which are each mutated independently (b)
        to predict the next state.
        The measurement function $h$ allows each particle state
        to be mapped to a GPS coordinate (a),
        which can be compared to the observed location (c)
        to calculate each particle's likelihood to compute resampling weights.
    }
        % The vehicles state is estimated using a set of discrete points, or particles,
        % each of which is independently mutated to make predictions of the future state.
        % Multimodality is easily handled, as demonstrated here,
        % without degeneration of the filter.}
    \label{fig:pf_state}
\end{figure}

\afterpage{\clearpage}

In a particle filter, the posterior distribution of the state at time $t_{k-1}$
is represented by a set of discrete points, or particles, (Figure~\ref{fig:pf_state_prev}),
each with an associated weight $W_{k-1}^{(i)}$.
The state can then be expressed using the Dirac delta function $\dot\delta_X(x)$ \citep{cn},
so that
\begin{equation*}
p(\bX_{k-1} | \bY_{1:k-1}) \approx 
    \sum_{i=1}^N W_{k-1} \dot\delta_{\bX_{k-1}}(\bX_{k-1}^{(i)})
\end{equation*}
each of which is independently updated or \emph{mutated} using the transition function $f$ (Figure~\ref{fig:pf_state_predict})
\begin{equation*}
p(\bX_k | \bX_{k-1}) \approx 
    \sum_{i=1}^N W_{k-1} \dot\delta_{\bX_{k}}(f(\bX_{k-1}^{(i)}, \psi))
\end{equation*}
using parameter vector $\psi$ containing all of the necessary parameters
for the model (including system noise).
After mutation, 
the state is updated by reweighting the particles using the likelihood $p(\bY_k | \bX_k^{(i)})$ 
(Section~\ref{sec:pf_update}) and standardising so $\sum_{i=1}^N W_k^{(i)} = 1$,
\begin{equation*}
W_k^{(i)} = \frac{W_{k-1}^{(i)} p(\bY_k | \bX_k^{(i)})}{
    \sum_{j=1}^N W_{k-1}^{(j)} p(\bY_k | \bX_k^{(j)})
}
\end{equation*}
generating a posterior estimate of the state
\begin{equation*}
p(\bX_k | \bY_k) \approx  
    \sum_{i=1}^N W_{k} \dot\delta_{\bX_{k}}(\bX_{k}^{(i)})
\end{equation*}

One problem with particle filters is degeneration,
which occurs when the particle sample no longer approximates the target distribution
due to the particles becoming too diverse
and all of the weight falling onto only a small number of them.
This occurs naturally over time, but the rate at which it occurs depends 
on a relationship between how well the transition function predicts future states,
and the variance of those predictions (system noise).
To avoid degeneration,
particle filters use \emph{importance resampling}
using importance weights $\{W_k^{(1)}, \ldots, W_k^{(N)}\}$,
which removes improbable particles and replaces them with more probable ones,
as demonstrated in Figure~\ref{fig:pf_state_predict2}.
However, resampling requires sorting $N$ particles,
which is of $\mathcal{O}(N\log N)$ complexity,
so we can reduce computational costs by only resampling when required.
To prevent degeneration of the particle filter state,
the sample is resampled whenever the
effective sample size $N_{\text{eff}} = 1 / \sum_i (W_k^{(i)})^2$
falls below a specified threshold $N_{\text{thres}}$.
We used a fixed threshold value of $N_{\text{thres}} = N/10$
for the current work.
\textbf{Check this}.

\subsubsection{Vehicle transition function}
\label{sec:pf_prediction}

The particle filter allows us to flexibly model bus behaviour,
which in our current model includes
\begin{itemize}
\item non-constant speed along roads (between observations), and
\item bus stop behaviour, which involves optional stopping and waiting times
    while passengers board and disembark.
\end{itemize}
To handle the above, the transition function consists of two models,
one for each of the above behaviours.
The first models the speed and distance travelled of the vehicle 
by using Newton's Laws of Motion,
which, letting $\delta = t_k - t_{k-1}$ and $\epsilon_k^{(i)}\sim\mathrm{N}(0, \sigma^2)$, is
\begin{equation}
\label{eq:trans_newton}
\bX_k^{(i)} = f_1(\bX_{k-1}^{(i)}, \delta_k, \epsilon_k^{(i)}) = 
    \begin{bmatrix}
        x_{k-1}^{(i)} + \delta(\dot x_{k-1}^{(i)} + \epsilon_k^{(i)}) \\
        \dot x_{k-1}^{(i)} + \epsilon_k^{(i)}
    \end{bmatrix}.
\end{equation}

The second vehicle behaviour requires a model of bus stop wait times.
We assume the vehicle is approaching stops $j$,
and will stop with some probability $\pi_j$.
If it does not stop, the remaining travel time $\delta_k$ is unaltered;
otherwise, the bus waits for a minimum of $\gamma$ seconds---this 
accounts for deceleration, opening and closing of doors, and acceleration
\citep{Hands_2015}---and then waits while passengers board and disembark,
which is assumed to be exponentially distributed with mean $\tau_j$.
This leads to the following model,
\begin{equation}
\label{eq:trans_stop}
\begin{split}
p_j^{(i)} &\sim \mathrm{Bernoulli}(\pi_j) \\
\tilde d_j^{(i)} &\sim \mathrm{Exponential}(\tau_j) \\
d_j^{(i)} &= p_j^{(i)}(\gamma + \tilde d_j^{(i)}).
\end{split}
\end{equation}

The combination of (\ref{eq:trans_newton}) and (\ref{eq:trans_stop})
is handled by the transition function algorithm,
displayed in Algorithm~\ref{fig:algorithm}.
While (\ref{eq:trans_stop}) is used directly in the algorithm,
(\ref{eq:trans_newton}) is modified to allow the bus to approach a stop
and conditionally wait before continuing,
until $\delta_k$ reaches zero.
The algorithm also keeps track of each particle's travel time $z_\ell^{(i)}$
along each segment $\ell$ of the route,
storing the times so that the posterior distribution of the travel time
can be computed once all particles have completed segment $\ell$:
\begin{equation}
\label{eq:post_tt}
p(z_\ell | \bY_{1:k}) \approx
    \sum_{i=1}^N W_k^{(i)} \dot\delta_{z_\ell}(z_\ell^{(i)}).
\end{equation}



\renewcommand{\algorithmicrequire}{\textbf{Start:}}
\newcommand{\algorithmicbreak}{\textbf{break}}
\newcommand{\Break}{\State \algorithmicbreak}
\begin{algorithm}[t]
    \caption{Particle mutation function.}
    \label{fig:algorithm}
    \begin{algorithmic}[2]
    \Require $x_{k-1}, \dot x_{k-1}, \boldsymbol z$\\
    Initialisation $\delta \gets t_k - t_{k-1}$, 
        $j \gets stop\_index(x, \boldsymbol{S})$
    \If {$u \sim \mathrm{U}(0,1) < 0.5$ and $x = S_j$}
        \Comment{Remain longer at stop}
        \State $d_j \sim \mathrm{Exp}(1/\tau_j)$
        \State $\delta \gets \delta - \gamma - d_j$
    \ElsIf {$u \sim \mathrm{U}(0,1) < 0.1$}
        \Comment{Occasionally bus stopped due to traffic}
        \State $w \gets \min(\delta, \mathrm{Exp}(1/\delta))$
        \State $\delta \gets \delta - w$
        \State $z_j \gets z_j + w$
        \Comment{Congestion counts as travel time}
    \EndIf
    \While {$\delta > 0$ and $x < S_J$} 
        \State $\dot x \sim \mathrm{N}(\dot x, \sigma^2)$
        \Comment{System noise, truncated to be non-negative}
        \State $x \leftarrow x + \dot x$
        \If {$x \geq S_{j+1}$}
            \If {$j+1 = J$}
                \Comment{Break at end of route}
                \State $x \gets S_J$
                \Break
            \EndIf
            \State $u_j \sim \mathrm{U}(0,1)$
            \Comment{Simulate vehicle stopping}
            \If {$u_j < \pi$}
                \State $d_j \sim \mathrm{Exp}(1/\tau_j)$
                \State $\delta \gets \delta - \gamma - d_j$
                \State $x \gets S_j$
            \EndIf
            \State $j\gets j+1$
        \EndIf
        \State $\delta \gets \delta - 1$
        \State $z_j \gets z_j + 1$
        \Comment{Track travel time along road}
    \EndWhile
    \\
    \Return $x, \dot x, \boldsymbol z$
    \end{algorithmic}
\end{algorithm}

\afterpage{\clearpage}


\subsubsection{Updating state using the observation likelihood}
\label{sec:pf_update}

To compute the likelihood of the observation given each particle state,
$p(\bY_k|\bX_k^{(i)})$,
the measurement function $h$ is needed along with an
\emph{geographical projection} $g$ to allow comparison of $\bY_k$,
a GPS observation, with $\bX_k$, a distance travelled along the route in meters.

The measurement function $h$ computes the GPS coordinates by using the 
shape information provided by GTFS and travelling $x_k$ meters along 
this two-dimensional line.
Meanwhile, we use the \emph{Equirectangular projection} \citep{Snyder_1998},
$\br = g(\bY_1 | \bY_0)$,
so that the magnitude of $\boldsymbol{r}$ is equal to the physical distance
between $\bY_0$ and $\bY_1$, with both dimensions in meters.
Let $\bY_i = [\lambda_i, \psi_i]^\top$,
with $\lambda_i$ and $\psi_i$ measured in radians,  
and $R$ is the Earth's radius, in meters, then
\begin{equation}
\br = 
g(\bY_1 | \bY_0) = 
    \begin{bmatrix}
        r_1 \\ r_2
    \end{bmatrix} =
    R \begin{bmatrix}
        (\lambda_1 - \lambda_0)\cos\phi_0 \\
        \phi_1 - \phi_0
    \end{bmatrix}
\end{equation}
and, more importantly, the geographical distance $\mathcal{D}$ between the two 
GPS coordinates is given by
\begin{equation}
\label{eq:dist}
\mathcal{D}(\bY_0, \bY_1) = ||g(\bY_0 | \bY_1)|| = \sqrt{r_1^2 + r_2^2}
\end{equation}


Now we assume GPS observations have a known error of $\omega^2$ meters,
and that \mbox{$\br \sim \mathrm{N}(\boldsymbol{0}, \omega^2\mathbf{I})$},
which allows us to express $\br$ in terms of two independent
standard normal random variables $z_1, z_2 \sim \mathrm{N}(0,1)$,
which lets us express the distance between a vehicle observation $\bY_k$
and a state $\bX_k$ using (\ref{eq:dist}) and the measurement function $h$,
\begin{equation}
\label{eq:obs_dist}
\mathcal{D}(\bY_k, h(\bX_k)) = \sqrt{r_{k1}^2 + r_{k2}^2} 
    = \sqrt{(\omega z_1)^2 + (\omega z_2)^2}
    = \sqrt{\omega^2 (z_1^2 + z_2^2)}
\end{equation}

Finally, it is known that the distribution of the sum of two indendent 
standard normal variables is exponentially distributed with rate $\frac{1}{2}$,
and that if $Z \sim \mathrm{Exponential}(\theta)$ then
$cZ \sim \mathrm{Exponential}(\frac{\theta}{c})$,
so the \emph{squared} distance from (\ref{eq:obs_dist}) has exponential distribution
\begin{equation}
\label{eq:obs_exp}
\mathcal{D}(\bY_k, h(\bX_k))^2 =
\omega^2(z_1^2 + z_2^2) \sim \mathrm{Exponential}\left(\frac{1}{2\omega^2}\right).
\end{equation}

The likelihood of the observation $\bY_k$ given a state estimate $\bX_k^{(i)}$
can now be expressed using (\ref{eq:obs_exp}) and the PDF of the exponential distribution,
\begin{equation}
\label{eq:lhood}
p(\bY_k | \bX_k^{(i)}, \epsilon) =
\frac{1}{2\omega^2}\exp\left\{
    -\frac{\mathcal{D}(\bY_k, h(\bX_k^{(i)}))^2}{2\omega^2}
\right\},
\end{equation}
where the distance between two GPS coordinates is easily computed,
making (\ref{eq:lhood}) easy and fast to evaluate within the particle filter algorithm.


It is worth noting that this representation of the likelihood is only
possible due to the discrete nature of the particle filter state estimate;
in the Kalman filter, which has often been used in vehicle tracking,
the measurement \emph{matrix} is used, and required that there is a linear
transformation between the state and its observations.
To enable this, applications first estimate the \emph{observed distance travelled}
by snapping the GPS observations to the route,
which introduces unnecessary error and uncertainty into the model.
Our approach avoids this, which makes it more stable in locations where two 
parts of the route are close to each other,
such as at loops, where a single GPS observation might have two likely ``snapping'' points.


\subsection{Network model}
\label{sec:kf}

The primary objective of the network model is to estimate the \rt traffic conditions
(travel time) along roads in the transit network, 
as well as make short-term predictions for estimating arrival times.
In this paper, we model each road segment independently,
as including correlations simplifies the model to not only a one-dimensional Kalman filter,
but it also means computations can be run in parallel,
greatly increasing the real-time performance of the model.


The network state $\boldsymbol\theta_c = \{\theta_c^\ell\}_{\ell = 1}^L$ is the travel time 
of transit vehicles along road segment $\ell$ at time $t_c$,
of which observations $Z_c^{\ell m}$
are obtained from the particle filter for vehicle $m$ as defined in
(\ref{eq:post_tt}), such that 
(after adding the $m$ superscript to identify unique vehicles), 
\begin{equation*}
Z_c^{\ell m} = \mathrm{E}(p(z_c^{\ell m} | \bY^m_{1:k})).
\end{equation*}
The measurement error is also estimated from the particle filter,
\begin{equation*}
R_c^{\ell m} = \mathrm{Var}(p(z_c^{\ell m} | \bY^m_{1:k})).
\end{equation*}

The model of travel times becomes a simple reduction of (\ref{eq:rbe_model}) 
to the one dimensional case, 
with $f$ and $h$ both unity,
system noise $v_c^\ell \sim \mathrm{N}(0, \nu^2)$
with $\Delta_c = t_c - t_{c-1}$
and $\nu_\ell^2$ is the variance of travel times per second,
and the observation error $e_c^{\ell m} \sim \mathrm{N}(0, R_c^{\ell m})$:
\begin{equation*}
\begin{split}
\theta_c^\ell &= \theta_{c-1}^\ell + \Delta_c^\ell v_c^\ell \\
Z_c^{\ell m} &= \theta_c^\ell + e_c^{\ell m}
\end{split}
\end{equation*}


Since multiple vehicles can travel along a road simultaneously,
we used an information filter to implement the network model.
The information filter is a transformation of the Kalman filter in which the
\emph{information matrix} and \emph{information vector} are used in place of 
the covariance matrix and state vector, respectively.
This allows multiple observations to be added together to update the state
in a single iteration,
which is the case when multiple vehicles travel along a road at the same time.


The state at time $t_{c-1}$ is parameterised by its mean and variance,
conditional on all of the data from all vehicles up until time $t_{c-1}$,
$\hat \theta_{c-1|c-1}^\ell = \mathrm{E}(\theta_c^\ell | Z_{1:c-1}^{\ell\boldsymbol{\cdot}})$
and \mbox{$P_{c-1|c-1}^\ell = \mathrm{var}(\theta_{c-1}^\ell | Z_{1:c-1}^{\ell\boldsymbol{\cdot}})$}, respectively,
which are predicted from the previous state estimate using the predictive model
\begin{align*}
\label{eq:kf_transition}
\hat \theta^\ell_{c|c-1} &= \hat \theta^\ell_{c-1|c-1} \\
P^\ell_{c|c-1} &= P^\ell_{c-1|c-1} + (\Delta_c^\ell \nu_c^\ell)^2
\end{align*}

For the update step, the parameters are transformed into an information
space parameterised by the information matrix $U^\ell_c = (P_{c|c-1})^{-1}$
and the information vector $u^\ell_c~=~\hat \theta^\ell_{c|c-1} (P^\ell_{c|c-1})^{-1}$.
The travel time estimate of vehicle $m$ along segment $\ell$,
$Z_c^{\ell m}$, along with its uncertainty $R^{\ell m}_c$,
are transformed to a measurement information covariance matrix 
$B_c^{\ell m}~=~(R^{\ell m}_c)^{-2}$
and measurement information vector $b_c^{\ell m}~=~Z^{\ell m}_c (R^{\ell m}_c)^{-2}$.
The total information is the sum of the information over all $M_c^\ell$ vehicles
that traversed segment $\ell$ since the last update,
so the update equation is
\begin{align*}
U^\ell_{c|c} &= U^\ell_{c|c-1} + \sum_{m=1}^{M_c^\ell} B^{m\ell}_{c} \\
\hat u^\ell_{c|c} &= \hat u^\ell_{c|c-1} + \sum_{m=1}^{M_c^\ell} b^{m\ell}_{c}.
\end{align*}
The desired parameter estimates are obtained 
by the inverse transformations
\begin{equation*}
\hat \theta^\ell_{c|c} = \frac{\hat u^\ell_{c|c}}{U^\ell_{c|c}} 
\quad\text{and}\quad
P^\ell_{c|c} = \frac{1}{U^\ell_{c|c}}
\end{equation*}
which can now be used in the prediction of travel times
for upcoming buses. 



