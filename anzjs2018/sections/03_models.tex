\section{The Models}
\label{sec:models}

\subsection{Real-time vehicle model}
\label{sec:pf}

The vehicle model involves infering the \emph{unobservable state} $\bX_k$ 
of a vehicle at time $t_k$ from the \emph{observed state} $\by_k$,
which in this case is the GPS location of the vehicle,
where $\lambda_k$ and $\phi_k$ are the longitude and latitude, respectively.
\begin{equation}
\label{eq:vehicle_observation}
\by_k = \begin{bmatrix} \lambda_k \\ \phi_k \end{bmatrix}
\end{equation}
To construct the unobservable state, 
each vehicle is assumed to follow the trajectory defined by a trajectory function of time, $x(t)$,
such that the distance traveled along a route by a vehicle at time $t_k$ is given by
$x_k = x(t_k)$.
Additionally, speed and acceleration are given by
the first and second derivatives of the trajectory function,
$\dot x_k = x'(t_k)$ and $\ddot x_k = x''(t_k)$, respectively.
Exactly how the state is constructed will depend on the model being used.


Since observations of the vehicle are irregular, 
a \emph{transition function} $f$ is used to
predict the vehicle's state at time $t_k$ given the current knowledge of the state at time $t_{k-1}$,
as well as accounting for \emph{system noise}, $\sigma_x^2$.
We also define a \emph{measurement function} $h$ which describes the relationship between
the observable and unobservable states,
and accounting for \emph{measurement error}, $\bv_k$,
\begin{equation}
\label{eq:vehicle_measurement}
\by_k = h(\bX_k) + \bv_k = h(x_k) + \bv_k,
\end{equation}
so all models need to estimate at least $x_k$, the distance traveled along the route.


The model is split into two steps, prediction and update.
The prediction involves using the transition function to predict
the next state, $\hat\bX_{k|k-1}$,
while the update step using a likelihood function and the measurement function
to update the state to account for the observed vehicle location, $\by_k$,
giving a final estimate of $\hat\bX_{k|k}$.


\subsubsection{Predicting state using the transtion function}
\label{sec:pf_prediction}

The transition function is a model of bus behaviour that allows
predictions of future states to be made given the most recently 
updated state.
Developing a good model for bus behaviour allows us to estimate
parameters of interest, such as vehicle speed,
dwell time at bus stops, and travel time along individual roads,
the last of which is of course our primary objective.
In this section, we propose [HOW MANY?] increasingly complex models
of bus behaviour.


%% MODEL 1
The first model, $\bM_0$, assumes a constant speed between observations
using the state
$\bX_k = \left[\begin{smallmatrix}x_k && \dot x_k\end{smallmatrix}\right]^\top$
and Gaussian system noise with mean zero and variance adjusted by
$\Delta_k = t_k - t_{k-1}$.
\begin{equation*}
\hat\bX_{k|k-1} = f_0(\bX_{k-1|k-1}, \Delta_k, \sigma_x^2) =
\begin{bmatrix}
x_{k|k-1} & \dot x_{k|k-1}
\end{bmatrix}
\end{equation*}
where
\begin{align*}
x_{k|k-1} &= x_{k-1|k-1} + \Delta_k \dot x_{k|k-1} \\
x_{k|k-1} &= \dot x_{k-1|k-1} + w_k \\
w_k &\sim \mathcal{N}_T(0, \Delta_k \sigma_x^2)
\end{align*}
Here, $\mathcal{N}_T$ denotes a truncated normal distribution such that
the system noise is truncated to ensure the speed remains between 0 and 30~m/s.


While $\bM_0$ may perform well on average,
and indeed if all we were interested in was estimating $x_k$ it may be adequate;
however, it is missing two important features of bus behaviour:
\begin{itemize}
\item variable speeds along roads, \emph{which is what we are truly interested in}
\item unknown dwell times at bus stops
\end{itemize}


The first of these is incorporated into $\bM_1$,
which allows vehicle speeds to vary over time.
This model uses the same state as $\bM_0$,
but a different transition function $f_1$.
In this case, the state equations depend on a sequence of system noise,
$\bw_k = \{\bw_k^1, \cdots, \bw_k^{\Delta_k}\}$
\begin{align*}
x_{k|k-1} &= x_{k-1|k-1} + \Delta_k \dot x_{k-1|k-1} + \sum_{i=1}^{\Delta_k} w_k^i \\
\dot x_{k|k-1} &= \dot x_{k-1|k-1} + \sum_{i=1}^{\Delta_k} w_k^i \\
w_k^i &\sim \mathcal{N}_T(0, \sigma_x^2)
\end{align*}
Again, system noise is assumed to have a Gaussian distribution, 
truncated to ensure that the vehicle's speed is always between 0 and 30~m/s.
This is achieved by iteratively estimating each second the the vehicle's state,
each time ensuring the speed remains in the specified range.



\subsection{Network model}
\label{sec:kf}

Two parts to this - first, the prior prediction step; then, the data update step.
