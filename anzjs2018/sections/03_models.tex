\section{The Models}
\label{sec:models}

The real-time nature of this application (means that for) both
the vehicle and network models a recursive Bayesian filtering approach is logical,
and indeed real-time tracking and robotics applications often use such (models).
In each of the following models, 
we assume an underlying Markov process with state $\mathcal{X}_k$,
of which observations $\mathcal{Y}_k$ are made,
which leads us to the following general model,
\begin{equation}
\label{eq:rbe_model}
\begin{split}
\mathcal{X}_k &= f(\mathcal{X}_{k-1}, \omega_k) \\
\mathcal{Y}_k &= h(\mathcal{X}_k, \nu_k)
\end{split}
\end{equation}
where $\omega_k\sim\mathrm{N}(0,\mathcal{Q})$ is the system noise,
and $\nu_k\sim\mathrm{N}(0,\mathcal{R})$ is measurement error.
The transition function $f$ determines the relationship between consecutive states,
while the measurement function $h$ is a deterministic function describing the
relationship between the underlying and observed states.

The following sections describe two Bayesian filter models based on (\ref{eq:rbe_model}) used in this application.
The first model is implemented using a particle filter to estimate vehicle states,
with the primary objective of estimating vehicle travel times along roads,
while the second using these estimated times to update road states,
and is implemented using a Kalman filter.

\subsection{Real-time vehicle model}
\label{sec:pf}

The underlying vehicle state at time $t_k$ consists of
the vehicle's distance traveled $x$ in meters along the route and
its speed $\dot x$ in meters per second.
Additionally, we include travel times along road segments along the route, 
$\bz = (z_1,\ldots,z_L)^\top$, in seconds,
which are independent of time but included in the state as they are estimated
sequentially as the vehicle traverses the route.
These are combined into the state vector,
\begin{equation}
\label{eq:vehicle_state}
\bX_k = 
\begin{bmatrix}
    x_k \\ \dot x_k \\ \bz
\end{bmatrix}.
\end{equation}
Observations of the vehicle are made at time $t_k$ using a GPS,
giving the longitude $\lambda_k$ and latitude $\phi_k$ of the vehicle,
giving us the observation vector
\begin{equation}
\label{eq:vehicle_obs}
\bY_k = \begin{bmatrix} \lambda_k \\ \phi_k \end{bmatrix}.
\end{equation}


We decided to employ a particle filter to estimate $\bX_k$,
since it is a highly flexible approach and has been used in recent 
transit vehicle modeling applications \citep{Hans_2015}.
Our main justification for using it is that it handles multimodality very well,
which is a common feature of the proposal distribution particularly around bus stops.
Another advantage is the intuitive likelihood function, 
which is described later.
Conversely, the particle filter is a computationally demanding method,
as thousands of particles are required per bus.
Section~\ref{sec:rt} describes how we were able to implement the particle filter in real-time.


In the particle filter, the posterior distribution of the state at time $t_k$,
is estimated by a set of discrete points, or particles,
\begin{equation}
p(\bX_k | \bY_k) \approx \tilde\bX_{k|k} := (\bX_k^{(i)})_{i=1}^N
\end{equation}
each of which is independently updated or \emph{mutated} using the transition function $f$,
\begin{equation}
p(\bX_k | \bX_{k-1}) \approx \tilde\bX_{k|k-1} := 
\left(f(\bX_{k-1}^{(i)}, \psi)\right)_{i=1}^N
\end{equation}
using a parameter vector $\psi$ containing all of the necessary parameters
for the model (including system noise).
After mutation, \emph{selection} of a new set of particles is performed by
importance resampling using likelihood weights.


\subsubsection{Vehicle transition function}
\label{sec:pf_prediction}
As we are using a particle filter to implement the vehicle model,
the transition function $f$ can be a complex description of bus behaviour,
most notably
\begin{itemize}
\item non-constant speed along roads (acceleration as system noise),
\item stopping and waiting at bus stops while passengers board and disembark, and
\item stopping and waiting at intersections.
\end{itemize}

For each particle, the transition function generates a plausible trajectory,
using system noise parameter $\sigma^2$ which describes 
how the vehicles acceleration changes as a random process.
% In its simplest form, the transition function becomes,
% from Newton's laws of motions,
% \begin{equation}
% \begin{split}
%     x_k &= x_{k-1} + \delta_k \dot x_{k-1} \\
%     \dot x_k &= \dot x_{k-1} + w_k,
%     \quad w_k \sim \mathrm{N}(0, \delta_k \sigma^2)
% \end{split}
% \end{equation}
However, the vehicle does not simply travel constantly along the route,
as it needs to service bus stops along the way.
Therefore, the transition function includes 
stopping probabilities $\boldsymbol\pi = (\pi_1,\ldots,\pi_J)^\top$ at bus stops,
dwell times $\boldsymbol\tau = (\tau_1,\ldots,\tau_J)^\top$ for passengers to
board and disembark (conditional on the bus stopping),
and the minimum dwell time at stops, $\gamma$.
For these parameters, we used constant values for all stops,
and based the values on those used by \cite{Hans_2015};
future work will look at modeling these separately in real-time also.
Also in this application, due to the complexity of the problem,
we use bus stops to define road segments.
Therefore no intersection model is required,
only the segment travel times as predicted by the network model at time $t_k$,
$\btheta(t_k) = (\theta_1(t_k), \ldots, \theta_L(t_k))^\top$
(see section~\ref{sec:kf}).
Future work will adapt this to use intersection that are indpendent 
of bus stops, and therefore completely independent of routes,
which will allow increasing the complexity of the model to incorporate
intersection stopping probabilities and waiting times.
The details of the transition function are given in the algorithm defined in the Appendix,
which implements those features discussed above.



\subsubsection{Updating state using the observation likelihood}
\label{sec:pf_update}

After mutating the particle set, the posterior distribution is obtained by
\emph{selection}, which is implemented by importance resampling using
likelihood weights.
To do so, we need to define the measurement function $h$,
and an additional function $g$ which transforms GPS coordinates onto a flat
surface (the geographical equirectangular projection,
which allows us to place a bivate normal likelihood on the data.
The model for the observation generation is,
assuming GPS error $\epsilon^2$,
\begin{equation}
\label{eq:pf_obs_model}
g(\bY_k) = g(h(\bX_k)) + \br_k,
\quad \br_k \sim \mathrm{N}(\boldsymbol{0}, \epsilon^2\boldsymbol{I})
\end{equation}

% The vehicle model involves infering the \emph{unobservable state} $\bX_k$ 
% of a vehicle at time $t_k$ from the \emph{observed state} $\by_k$,
% which in this case is the GPS location of the vehicle,
% where $\lambda_k$ and $\phi_k$ are the longitude and latitude, respectively.
% \begin{equation}
% \label{eq:vehicle_observation}
% \by_k = \begin{bmatrix} \lambda_k \\ \phi_k \end{bmatrix}
% \end{equation}
% To construct the unobservable state, 
% each vehicle is assumed to follow the trajectory defined by a trajectory function of time, $x(t)$,
% such that the distance traveled along a route by a vehicle at time $t_k$ is given by
% $x_k = x(t_k)$.
% Additionally, speed and acceleration are given by
% the first and second derivatives of the trajectory function,
% $\dot x_k = x'(t_k)$ and $\ddot x_k = x''(t_k)$, respectively.
% Exactly how the state is constructed will depend on the model being used.


% Since observations of the vehicle are irregular, 
% a \emph{transition function} $f$ is used to
% predict the vehicle's state at time $t_k$ given the current knowledge of the state at time $t_{k-1}$,
% as well as accounting for \emph{system noise}, $\sigma_x^2$.
% We also define a \emph{measurement function} $h$ which describes the relationship between
% the observable and unobservable states,
% and accounting for \emph{measurement error}, $\bv_k$,
% \begin{equation}
% \label{eq:vehicle_measurement}
% \by_k = h(\bX_k) + \bv_k = h(x_k) + \bv_k,
% \end{equation}
% so all models need to estimate at least $x_k$, the distance traveled along the route.


% The model is split into two steps, prediction and update.
% The prediction involves using the transition function to predict
% the next state, $\hat\bX_{k|k-1}$,
% while the update step using a likelihood function and the measurement function
% to update the state to account for the observed vehicle location, $\by_k$,
% giving a final estimate of $\hat\bX_{k|k}$.


% \subsubsection{Predicting state using the transtion function}
% \label{sec:pf_prediction}

% The transition function is a model of bus behaviour that allows
% predictions of future states to be made given the most recently 
% updated state.
% Developing a good model for bus behaviour allows us to estimate
% parameters of interest, such as vehicle speed,
% dwell time at bus stops, and travel time along individual roads,
% the last of which is of course our primary objective.
% In this section, we propose [HOW MANY?] increasingly complex models
% of bus behaviour.


% %% MODEL 1
% The first model, $\bM_0$, assumes a constant speed between observations
% using the state
% $\bX_k = \left[\begin{smallmatrix}x_k && \dot x_k\end{smallmatrix}\right]^\top$
% and Gaussian system noise with mean zero and variance adjusted by
% $\Delta_k = t_k - t_{k-1}$.
% \begin{equation*}
% \hat\bX_{k|k-1} = f_0(\bX_{k-1|k-1}, \Delta_k, \sigma_x^2) =
% \begin{bmatrix}
% x_{k|k-1} & \dot x_{k|k-1}
% \end{bmatrix}
% \end{equation*}
% where
% \begin{align*}
% x_{k|k-1} &= x_{k-1|k-1} + \Delta_k \dot x_{k|k-1} \\
% x_{k|k-1} &= \dot x_{k-1|k-1} + w_k \\
% w_k &\sim \mathcal{N}_T(0, \Delta_k \sigma_x^2)
% \end{align*}
% Here, $\mathcal{N}_T$ denotes a truncated normal distribution such that
% the system noise is truncated to ensure the speed remains between 0 and 30~m/s.


% While $\bM_0$ may perform well on average,
% and indeed if all we were interested in was estimating $x_k$ it may be adequate;
% however, it is missing two important features of bus behaviour:
% \begin{itemize}
% \item variable speeds along roads, \emph{which is what we are truly interested in}
% \item unknown dwell times at bus stops
% \end{itemize}


% The first of these is incorporated into $\bM_1$,
% which allows vehicle speeds to vary over time.
% This model uses the same state as $\bM_0$,
% but a different transition function $f_1$.
% In this case, the state equations depend on a sequence of system noise,
% $\bw_k = \{\bw_k^1, \cdots, \bw_k^{\Delta_k}\}$
% \begin{align}
% \label{eq:transition_f1}
% x_{k|k-1} &= x_{k-1|k-1} + \Delta_k \dot x_{k-1|k-1} + \sum_{i=1}^{\Delta_k} w_k^i \nonumber \\
% \dot x_{k|k-1} &= \dot x_{k-1|k-1} + \sum_{i=1}^{\Delta_k} w_k^i \\
% w_k^i &\sim \mathcal{N}_T(0, \sigma_x^2) \nonumber
% \end{align}
% Again, system noise is assumed to have a Gaussian distribution, 
% truncated to ensure that the vehicle's speed is always between 0 and 30~m/s.
% This is achieved by iteratively estimating each second the the vehicle's state,
% each time ensuring the speed remains in the specified range.


% The next model, $\bM_2$, allows for dwell times at bus stops.
% Since buses can either stop and wait, or not stop,
% multimodality is introduced to the system,
% which requires a modeling framework, such as the particle filter used here,
% to handle it.
% For each bus stop $j = 1, \ldots, M$ along a route,
% its distance into the trip, $S_j^d$, 
% is used to determine when a vehicle passes a stop.
% When this occurs, the dwell time $\delta_j \geq 0$ at stop $j$ is modelled
% using the stopping probability, $\pi_j$,
% a common dwell time parameter $\gamma$ which models the
% deceleration, opening and closing of doors, and acceleration at bus stops,
% and a stop dwell time $\tau_j$ for passengers alighting and disembarking.
% Therefore the stop dwell time is
% \begin{equation}
% \label{eq:dwell_time}
% \delta_j = p_j(\gamma + \tilde\delta_j),\quad
% p_j \sim \mathrm{Bern}(\pi_j),\quad
% \tilde\delta_j \sim \mathcal{E}(\tau_j)
% \end{equation}

% To define the transition function for this model, $f_2$,
% we need to account for multiple dwell times.
% Let $s_n$ be the index of the next stop,
% and sample dwell times for each upcoming stop $j = s_n, \ldots, M$,
% noting that these can be zero.
% Assuming the vehicle passes stops $s_n$ up to $J-1$,
% does it reach stop $J$ in the time remaining?
% \begin{equation}
% \tilde X_k^J = \min\left\{
% S_J^d, f_1(\hat X_{k|k-1}, \Delta_k - \sum_{j=s_n}^{J-1}\delta_j)\right\}
% \end{equation}
% Taking the maximum of the sequence gives the predicted distance
% traveled by the vehicle, so
% \begin{equation}
% \hat X_{k|k-1} = f_2 (\hat X_{k-1|k-1}, \Delta_k) =
% \max_{J = s_n, \ldots, M}\left\{
%     \tilde X_k^J
% \right\}
% \end{equation}

% Care needs to be made when generating the sequence,
% so in practice the state will be calculated recursively to ensure
% the system noise is consistent.
% \begin{equation*}
% \hat{\tilde X}_k^J =
% \begin{cases}
%     f_1(\hat{\tilde X}_k^{J+1}, \delta_J) & J < M \\
%     \hat X_{k-1|k-1} & J = M
% \end{cases}
% \end{equation*}

% The final model $\bM_3$ demonstrated here will allow acceleration to vary.
% \begin{align*}
% \label{eq:model_m3}
% x_{k|k-1} &= x(t_k) \\
% \intertext{where states are recursively calculated back to $t_{k-1}$}
% x(t) &= x(t - 1) + \dot x(t) \\
% \dot x(t) &= \dot x(t-1) + \ddot x(t) \\
% \ddot x(t) &= \ddot x(t-1) + w_t,\quad
% w_t \sim \mathcal{N}(0,\sigma_x^2)
% \end{align*}
% By truncating the acceleration, 
% the speed of the vehicle can be maintained between 0 and 30~m/s.
% Additionally, when approaching a bus stop that will be stopped at,
% the acceleration will be trunctated to below zero,
% however no physical deceleration will be modeled as this is too complex
% and stopping behaviour of buses can be sudden 
% (for example when passengers signal late that they wish to board or disembark).
% After servicing a stop, the bus will need to accelerate up to speed again. 


% \subsubsection{Updating state using the observation likelihood}
% \label{sec:pf_update}

% The update step involves comparing the predicted state $\bX_{k|k-1}$
% with the observed state $\bY_k$,
% using the measurement function $h$ to convert between the two state spaces,
% and a likelihood function.
% The measurement function is a deterministic function that uses only the 
% distance traveled, $x_k$, along a known path $\bR$ to obtain a GPS coordinate.
% The inverse measurement function, which is used in other approaches,
% is \emph{not deterministic}, and instead uses an optimisation to
% find the ``estimated'' distance traveled,
% which invariably leads to issues in the implementation.
% By using a particle filter approach, we avoid using $h^{-1}$
% and instead determine the liklihood using $\bY_k$ directly.

% For the likelihood, the GPS error is assumed symmetrical about the true
% position of the vehicle, with a radius of $\sigma_y^2$ meters.
% Define a projection function $g(\cdot | \bY)$ that projects points
% onto a cartesian surface (``Flat Earth''),
% so that the distance between points is the real Earth distance.
% Then the distance between the points is
% \begin{align*}
% \delta_k &= \sqrt{\left(g(h(X_k))_1 - g(\bY_k)_1\right)^2 +
%     \left(g(h(X_k))_2 - g(\bY_k)_2\right)^2} \\
% \intertext{which can be standardised by the GPS error}
% \tilde \delta_k &=
% \sqrt{\left(\frac{g(h(X_k))_1 - g(\bY_k)_1}{\sigma_y}\right)^2 +
%     \left(\frac{g(h(X_k))_2 - g(\bY_k)_2}{\sigma_y}\right)^2} =\frac{\delta_k}{\sigma_y} \\
% \intertext{Since we assume symmetric variance, this is actually}
% \tilde\delta_k &= \sqrt{Z_1^2 + Z_2^2}, \quad
% Z_1,Z_2 \sim \mathcal{N}(0,1)
% \end{align*}
% The sum of two normal random variables has an Exponential distribution with rate $\frac{1}{2}$,
% therefore 
% \begin{equation}
% \label{eq:lhood_exp}
% \left(\frac{\delta_k}{\sigma_y}\right)^2 \sim \mathrm{Exp}\left(\frac{1}{2}\right)
% \end{equation}

% Combining this together, we use a distance function $d(\bY_1, \bY_2)$ that calculates
% the geographic distance between two GPS coordinates,
% the likelihood is
% \begin{equation}
% \label{eq:lhood}
% f(\bY_k | \hat\bX_{k|k-1}, \sigma_y^2) =
% \frac{1}{2} \exp\left\{-\frac{d(\bY_k, h(\hat\bX_{k|k-1}))^2}{2\sigma^2}\right\}
% \end{equation}


\subsection{Network model}
\label{sec:kf}

% The second part of this appraoch involves a network model of transit network travel times along links in the network.
% Reliable ETAs will require a historical data component,
% used as a prior in the absense of data as well as for
% predicting future states, 
% and a realtime update component where the travel times of vehicles
% are used to update the current estimate of network state.

% The model will be another recursive Bayesian model as used for the vehicle state,
% as this provides a simple way to model realtime data.
% Since travel times can safely be assumed to be unimodal,
% we use an information filter (IF),
% which is a transformation of the Kalman filter
% using information instead of covariance,
% and allows the simple combination of multiple observations,
% for example when multiple buses traverse a road in the same update.
% The prediction step will involve a transition function that incorporates historical data
% and the update step will weight between observed and predicted travel times.

% The network state $\Theta_c = \{\theta_c^j\}_{j = 1}^J$ is the travel time 
% of transit vehicles along road segment $j$ at time $t_c$.
% In this implementation, we are assuming independence between each segment
% to provide a simple basis for the model; of course in reality this
% is not true and future work will investigate ways of incorporating this.

% Using an IF requires a transition matrix, $\mathbf{F}_c$;
% however, in this case the travel time model is constant so no
% transition matrix is required (i.e., $\mathbf{F}_c = 1$).
% Let $\Delta_c = t_c - t_{c-1}$ and $P_c^j$ the state uncertainty
% with system noise $Q_c = \Delta_c \sigma_b^2$, then the update equations are
% simply the normal KF update equations,
% \begin{align}
% \label{eq:kf_transition}
% \hat\theta^j_{c|c-1} &= \hat\theta^j_{c|c-1} \\
% P^j_{c|c-1} &= P^j_{c-1|c-1} + Q_c
% \end{align}

% For the update step, however, we need to transform into an information
% space parameterised by the information matrix 
% $\mathbf{Z}^j_{c|c-1} = P_{c|c-1}^{-1}$ 
% and the information vector $\mathbf{\hat z}^j_{c|c-1} = \hat\theta_{c|c-1} P_{c|c-1}^{-1}$.
% The measurement data obtained from the vehicle model are
% the travel time of vehicle $m$ along segment $j$,
% $\bar b^m_j$, and the uncertainty $s^m_j$.
% These can be transformed to a measurement information covaraince matrix
% and vector using the measurement matrix $\mathbf{H} = 1$ since,
% in this case, the observed state is the underlying state we are estimating (travel time).
% \begin{equation}
% I^m_{jc} = \frac{1}{(s^m_j)^{2}}\quad\text{and}\quad
% i^m_{jc} = \frac{\bar b^m_j}{(s^m_j)^2}
% \end{equation}
% The information update is now the sum of the information for all vehicles
% that traversed the segment since the last update, so
% \begin{align*}
% \mathbf{Z}^j_{c|c} &= \mathbf{Z}^j_{c|c-1} + \sum_{m=1}^M I^m_{jc} \\
% \mathbf{\hat i}^j_{c|c} &= \mathbf{\hat z}^j_{c|c-1} + \sum_{m=1}^M i^m_{jc}
% \end{align*}
% which are easily transformed back to get the desired state parameters.
% \begin{equation}
% \hat\theta^j_{c|c} = \frac{\mathbf{\hat z}^j_{c|c}}{\mathbf{Z}^j_{c|c}} 
% \quad\text{and}\quad
% P^j_{c|c} = \frac{1}{\mathbf{Z}^j_{c|c}}
% \end{equation}



