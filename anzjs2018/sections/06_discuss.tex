\section{Discussion and Future Work}
\label{sec:discussion}

In this paper we have described an approach to transit vehicle modeling
that enables the \rt estimation of road state from vehicle position data,
which will in future be used to make what will hopefully be more reliable 
arrival time predictions.
The vehicle model, implemented using a particle filter,
allows complex bus behaviours to be modeled,
with future work planned to explore the other parameters---dwell time and stopping probabilities---%
as well as to extend the model to also account for physical intersections.
The travel times estimated from the particle filter then allow modeling of
\rt road state.


The network model presented here is simple,
but provides the foundation for future work which will involve
developing a formal model that combines \rt information with
historical data, allowing for more accurate predictions of future road state,
for example incorporating peak time congestion forecasts.
This in turn will provide the basis for an arrival time prediction model,
a combination of vehicle state and road state forecasts.


The primary purpose of this paper was to present our approach and 
explore and demonstrate tis faesibility as a \rt option for RTI in transit systems.
We have been able to construct an application that runs in \rt and capable
of producing estimates within 30~seconds---although this can be reduced 
by using fewer particles or, if necessary, increasing the number of cores.


% The main goal of this work was to explore the faesibility
% of a particle filter as a \rt estimation method for transit vehicle states,
% and the use of estimated states to determine traffic conditions
% throughout the network.
% This allows vehicles to inform others---irrespective of route---of 
% current \rt traffic conditions,
% and provide a means of obtaining more reliable ETAs.


% We have shown that the particle filter a viable \rt option in terms of speed,
% and while it occasionally loses the bus,
% a lot of this is attributed to invalid data (incorrect route, direction, etc).
% Future development will incorporate more accurate stopping behaviour,
% perhaps using GTFS arrival and departure times in the likelihood function.


% The network model is still in the early stages of development,
% but given that the implementation is running well within our 30~second target,
% there is room for developing a more complex model.
% Future work will explore the use of historical data to develop informative priors
% for travel times,
% which will provide a useful fallback in the case of no data,
% or for making more accurate long-term predictions.
% Once such a model has been developed, we plan to explore prediction estimates,
% particuarly comparing point and intervals as a means of communicating 
% uncertainty to commuters.
