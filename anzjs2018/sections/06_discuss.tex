\section{Discussion and Future Work}
\label{sec:discussion}

The main goal of this work was to explore the faesibility
of a particle filter as a \rt estimation method for transit vehicle states,
and the use of estimated states to determine traffic conditions
throughout the network.
This allows vehicles to inform others---irrespective of route---of 
current \rt traffic conditions,
and provide a means of obtaining more reliable ETAs.


We have shown that the particle filter a viable \rt option in terms of speed,
and while it occasionally loses the bus,
a lot of this is attributed to invalid data (incorrect route, direction, etc).
Future development will incorporate more accurate stopping behaviour,
perhaps using GTFS arrival and departure times in the likelihood function.


The network model is still in the early stages of development,
but given that the implementation is running well within our 30~second target,
there is room for developing a more complex model.
Future work will explore the use of historical data to develop informative priors
for travel times,
which will provide a useful fallback in the case of no data,
or for making more accurate long-term predictions.
Once such a model has been developed, we plan to explore prediction estimates,
particuarly comparing point and intervals as a means of communicating 
uncertainty to commuters.
