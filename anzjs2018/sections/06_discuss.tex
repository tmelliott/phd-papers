\section{Discussion and Future Work}
\label{sec:discussion}

The main goal of this work was to explore the faesibility
of a particle filter as a \rt estimation method for transit vehicle states,
and the use of estimated states to determine traffic conditions
throughout the network.
This allows vehicles to inform others---irrespective of route---of 
how long it will take to reach future stops.


Not only is the particle filter a viable choice in terms of speed,
it has shown it can handle most situations without losing the vehicle,
with the exception of when the vehicle is not traveling along
the specified route.
Hopefully with future development,
the stopping behaviour can be more accurately modeled,
perhaps incorporating trip updates into the likelihood function.


The network model is still in the early stages of development,
but given that the implementation is running quickly---twice as fast as our target---%
there is room for developing a more complex model.
Future work will explore the use of historical data to develop informative priors
for travel times,
which will provide a useful fallback in the case of no data,
or for making more accurate long-term predictions.
Once such a model has been developed, we plan to explore prediction estimates,
particuarly comparing point and intervals as a means of communicating 
uncertainty to commuters.
