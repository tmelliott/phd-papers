\documentclass[times, doublespace]{anzsauth}
\usepackage{moreverb}
\usepackage{url}
\usepackage{grffile}
\usepackage{lineno}
\usepackage{lipsum}
\usepackage[UKenglish]{isodate}

\def\volumeyear{2018}
\linenumbers

\begin{document}
\cleanlookdateon
\runningheads{Short title of paper}{TOM~ELLIOTT AND THOMAS~LUMLEY}
\title{Long name of the paper}
\author{Tom Elliott\corrauth and Thomas Lumley}
\affiliation{University of Auckland}
\address{
    Department of Statistics, University of Auckland,\\
    Private Bag 92019, Auckland 1142, New Zealand\\
    Email: \texttt{tom.elliott@auckland.ac.nz}
}

\begin{abstract}
Predicting transit vehicle arrival time is a somewhat complex procress.
Road networks are dynamic, and can change from free-flowing to highly congested
very quickly.
Any reliable prediction framework needs to be able to respond to these changes,
but also to future trends, such as typical changes to traffic flow
at peak times, based on hisorical data.
Of course, the major constraint on any prediction framework is going to be
computational efficiency.
Due to the realtime nature of the problem, predictions need to be 
available as soon as possible after observing the vehicles' positions,
which are updated approximately every 30~seconds.
In this paper, we describe our C++ implementation of a particle filter to estimate
vehicle state in combination with a network state model,
allowing us to model transit vehicle flow through the network.
While our application here is being testing in Auckland, New Zealand,
we hope that our methods are general enough to be applied to any
transit network that uses GTFS.
Preliminary results show that, with suitable hardware,
the framework is fast enough while retaining the complexity necessary
to incorporate as much information as necessary to predict arrival times.
\end{abstract}

\keywords{particle filter; Kalman filter; transit modeling;
          transit networks; statistical computing; gtfs}

\maketitle
\section{Introduction}
\label{sec:intro}

The need for realtime information.
Where it goes wrong: 
- bad schedule calibration;
- no accounting for realtime traffic 
Literature review of historical and present methods?

Particle filter has proven useful to this kind of dynamic problem,
where the state distribution is non-gaussian, non-symmetric, 
multimodal...

Any other recent attempts at network-estimation?
There was a KF using point-to-point speed estimations? 

Computational aspect - historically, models had to be simple 
(e.g., KF) so they could run in realtime.
These days, running a somewhat powerful VM on a remote server
is almost standard practice, so we can take full advantage 
and use a more computationally demanding (particle filter)
but more accurate estimation technique - paired with a more complex model.

We first give an overview of the transit network,
and how we construct it from raw GTFS shape data.
This paper assumes a very basic model,
with a focus on the implementation in a realtime setting.




\section{Transit network construction}
\label{sec:gtfs}

Before we can usefully model buses in real-time, 
we need to construct a \emph{transit network}, 
consisting of \emph{intersections} (nodes),
connected by \emph{road segments} (edges).
In this way, each \emph{route} 
(a journey taken by a transit vehicle, from an origin to a destination by a fixed path)
can be represented as a sequence of road segments,
each of which we will model as vehicles travel along them (see section~\ref{sec:kf}).


\section{The Models}
\label{sec:models}

\subsection{Real-time vehicle model}
\label{sec:pf}

The real-time vehicle model that runs ... in real time?


\subsection{Network model}
\label{sec:kf}

Two parts to this - first, the prior prediction step; then, the data update step.


\section{Implementing in realtime}
\label{sec:rt}

How we implement it, choice of software (Rcpp = R + C++).
R: dealing with data structures is easier, maintainability, interfacing
C++: speed

Overall structure:
- load
- fetch positions
- initialize or mutate+update
- update network
- make ETA predictions (vehicle state + network state)

Some of the key things:
- minimise copy, parallelisation using OMP
- moving as much computation ``outside'' of the main loop as possible
  (e.g., ``pre''-predict vehicle/network states so only update required)
  so ETAs are generated ASAP after retrieving data
- keeping the GTFS database up-to-date by fetching new data each morning
- distribution - a cloud database vs a single protobuf file with everything 
  (maintenance/reliability/speed/size)


\section{Preliminary Results}
\label{sec:results}

Currently only have timings (i.e., yes this is plausible).
- Function of number of particles/number of cores/number of buses.
- Network coverage? i.e., how many road segments actually get enough data
to generate useful numbers

- any problems?


\section{Discussion and Future Work}
\label{sec:discussion}

It all seems to work (hopefully I don't need to change this D:)

Next steps include
- generating useful priors for ``fall back'' (i.e., no data, future) prediction
- develop and implement a more complex network update model (correlations, etc)
- prediction estimates (point vs interval)

\cite{Hans_2015}


\bibliographystyle{anzsj}
\bibliography{refs.bib}

\end{document}