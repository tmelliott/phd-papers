\documentclass[times, doublespace]{anzsauth}
\usepackage{moreverb}
\usepackage{url}
\usepackage{grffile}
\usepackage{lineno}
% \usepackage{lmodern}
\usepackage{lipsum}
\usepackage[UKenglish]{isodate}
\usepackage{float}
\usepackage{subcaption}
\usepackage{algorithm}
\usepackage{algpseudocode}
% \newfloat{lstfloat}{htbp}{lop}
% \floatname{lstfloat}{Listing}
% \def\lstfloatautorefname{Listing}

% \doublespace
\usepackage{afterpage}

\usepackage[acronym]{glossaries}
\newacronym[shortplural={ATIS}]{atis}{ATIS}{advanced traveller information system}
\newacronym{avl}{AVL}{automatic vehicle location}
\newacronym{apc}{APC}{automatic passenger counter}
\newacronym{rti}{RTI}{real-time information}
\newacronym{gps}{GPS}{Global Positioning System}
\newacronym{api}{API}{application programming interface}
\newacronym{gtfs}{GTFS}{general transit feed specification}
\newacronym{knn}{KNN}{$k$-nearest neighbour}
\newacronym{ann}{ANN}{artificial neural networks}
\newacronym{svm}{SVM}{support vector machines}
\newacronym{kf}{KF}{Kalman filter}
\newacronym{ekf}{EKF}{extended Kalman filter}
\newacronym{pf}{PF}{particle filter}
\newacronym{mcmc}{MCMC}{Markov chain Monte Carlo}
\newacronym[longplural={expected times of arrival}]{eta}{ETA}{expected time of arrival}
\newacronym{at}{AT}{Auckland Transport}
\newacronym{pt}{PT}{public transport}
\newacronym{ui}{UI}{user interface}
\newacronym{osm}{OSM}{OpenStreetMap}

\newcommand{\bv}{\boldsymbol{v}}
\newcommand{\bz}{\boldsymbol{z}}
\newcommand{\bw}{\boldsymbol{w}}
\newcommand{\bR}{\boldsymbol{R}}
\newcommand{\bT}{\boldsymbol{T}}
\newcommand{\br}{\boldsymbol{r}}
\newcommand{\btheta}{\boldsymbol{\theta}}

\newcommand{\rt}{real-time\ }

\def\volumeyear{2018}
\linenumbers

\begin{document}
\cleanlookdateon
\runningheads{Modeling transit vehicles in real-time}{TOM~ELLIOTT AND THOMAS~LUMLEY}
\title{A real-time particle filter for modeling travel times of transit vehicles through a GTFS-based road network}
\author{Tom Elliott\corrauth~and~Thomas~Lumley}
\affiliation{University of Auckland}
\address{
    Department of Statistics, University of Auckland, 
    Auckland 1142, New Zealand\\
    Email: \texttt{tom.elliott@auckland.ac.nz}
}
\ack{
    We would like to thank Auckland Transport for providing free,
    open access to the \rt data; and the Center for e-Research at the
    University of Auckland for providing and maintaining the virtual
    machine used in our work.
}

\begin{abstract}
Predicting the arrival time of a transit vehicle involves not only
knowledge of its current position and schedule adherence,
but also of traffic conditions along the remainder of the route.
Road networks are dynamic, 
and can change from free-flowing to highly congested very quickly,
which impacts the arrival time of transit vehicles,
particularly buses which often share the road with other vehicles,
so reliable predictions need to account for \rt and future traffic conditions.
The first step in this process is to construct a framework with which road state
(traffic conditions) can be estimated using \rt transit vehicle position data.
We propose a framework which implements a vehicle model using a particle filter
to estimate travel times along roads,
used by a second model to estimate \rt traffic conditions which can
later be used for arrival time prediction.
We demonstrate the \rt feasibility of such an approach in \rt,
and detail our implementation in R and C++.
Development and testing took place in Auckland, New Zealand,
with future work planned to incorporate historical data into the network model
to improve the reliability of ETAs.
\end{abstract}

\keywords{particle filter; transit modeling;
          transit networks; applications; gtfs}

\maketitle
\section{Introduction}
\label{sec:intro}


In public transport, \rt information (RTI)---%
most notably estimated times of arrival (ETAs)---%
keeps transport users informed of changes to their commute,
allowing them to plan their journey accordingly.
Previous research has shown that perceived waiting time is less
when arrival time information is available \citep{TCRP_2003b},
but here in Auckland, RTI is highly unreliable,
leading to frustration and ultimately detering public transport users.
Inaccurate ETAs are a major source of this frustration,
as they fluctuate---sometimes dramatically---over time.
This can be caused by complex changes to traffic conditions,
but equally can be due simply to poor schedule calibration.
Furthermore, buses are shown as \emph{on-time} 
if they are not present in the \rt system and have not been manually cancelled,
in which case ETAs are based solely on scheduled arrival times.
Once the scheduled arrival time has passed,
the service is removed from the \rt board,
leaving passengers unsure as to when---if at all---their bus will show up,
a phenomenon referred to as ``ghost buses'' by transport bloggers.


Arrival time prediction is only as reliable as the underlying model,
and while a lot of work has gone into developing public transport vehicle models
\citep{Cathey_2003,Jeong_2005,Yu_2011,Hans_2015},
many public transport providers---notably Auckland Transport---useno formal model is used.
Instead, ETAs are based solely on the scheduled arrival time
adjusted by the vehicle's arrival or departure delay at the most recent stop, 
\emph{if available},
the primary assumption being that the schedule is valid,
and secondly that there is no unusual congestion along the route;
neither of these are valid,
particularly in our test area of Auckland, New Zealand,
where infrastructure for buses (such as priority lanes) is limited,
and bus drivers, in general, 
do not actively adhere to the schedule.


A more robust modelling and prediction framework 
based on \rt congestion information would avoid making these assumptions.
Such a framework should consist of a robust vehicle model to estimate the position and speed
of transit vehicles from \rt GPS data,
and secondly a means of combining speed information from vehicles
to model traffic flows,
which can then be used to improve arrival time predictions.
Several vehicle modelling approaches were explored, 
including the Kalman filter \citep{Dailey_2001,Cathey_2003},
machine learning models \citep{Yu_2006,Chang_2010},
and the particle filter,
which has successfully been used in vehicle tracking applications
\citep{cn},
as well as in transport modelling \citep{Hans_2015}.
For several reasons, discussed further in Section~\ref{sec:pf},
the particle filter was used in the current application.s


The second component involves estimating traffic conditions throughout the transport network
using the information obtained from the vehicle model.
\cite{Yu_2010} improved prediction accuracy by using the travel times
of buses travelling along the same route.
A similar method presented by \cite{Hans_2015}
used headway, the time between consecutive vehicles at a point on the route,
as a predictor of travel time.
Since these approaches only work well on high frequency routes,
\cite{Yu_2011} showed further improvements by combining travel times 
from several routes;
however, this was limited to predefined converging routes.
In general, however, no comprehensive network modelling approach has been proposed using
solely GPS position data to model and account for congestion when estimating arrival times.


In Section~\ref{sec:models} of this paper, 
we describe the \rt vehicle and network state models
used to estimate traffic congestion using the transit vehicles
travelling through the network in real-time,
and in Section~\ref{sec:rt} we demonstrate the feasibility of using a particle filter
in \rt to model transit vehicles,
and evaluate its performance.
First, however, we descibe how we construct a
transit road network from GTFS data in Section~\ref{sec:gtfs}.




\section{Working with \rt transit data}
\label{sec:gtfs}

GTFS (general transit feed specification)
is an API (application programming interface) specification for transit data
detailing how it should be organised,
making access easier for application developers.
Developed and maintained by Google \citep{GoogleDevelopers_2006},
who make use of it in their Google Maps Transit Directions,
it is used by over 900~transit providers around the world,
including here in Auckland, New Zealand
(source \url{http://transitfeeds.com}).
An advantage of this standardised format is that,
provided an application depends solely on GTFS data,
after developing it locally in Auckland it can be deployed to any other GTFS-based
public transport system with minimal modification.


There are two components to GTFS.
The first, \emph{GTFS static}, includes information about
\begin{itemize}
\item \emph{stops}, a physical location where passengers can embark and disembark the vehicle;
\item \emph{routes}, a sequence of two or more stops displayed as a single service;
\item \emph{trips}, an instance of a route occuring at a specific time of day;
\item \emph{schedules}, specifying the arrival (and departure) times for each bus at each of its stops; 
\item \emph{shapes}, the sequence of points defining a vehicle's path along a route
\end{itemize}
\citep{GoogleDevelopers_2006}.
The second component is \emph{GTFS realtime},
which is only available in a subset of the providers due to the requirement of 
onboard GPS tracking devices and a central server.
It provides a standardised format for sharing vehicle positions and trip delays,
and are typically accessed by developers via an API can be used in \rt applications.

As mentioned in Section~\ref{sec:intro},
there are some major issues with the current ETA generation method.
These are almost directly attributed to GTFS-realtime trip updates,
which are currently the sole source of data for ETAs (in Auckland).
Trip updates are reported whenever a vehicle arrives or departs a stop,
and includes most importantly the delay between the scheduled and actual arrival times.
This delay is then propagated on to all future stops to adjust the ETA.
This assumes that the schedule is well calibrated and the time between scheduled arrivals
is representative of the real-world travel time between stops. 
Even more problematic is that, in the absense of trip updates,
the delay defaults to zero,
and unless the trip is manually cancelled,
appears on-time to passengers waiting at the stop.
If the bus is late or cancelled,
the trip disappears from the \rt board and passengers
are unable to distingush the two scenarios.


\subsection{Transit network construction}
\label{sec:network_build}

Arguably the most important predictor of arrival time is
the cumulative travel time along intermediate roads.
In most applications, however, this vital information is unavailable,
at least directly,
so here we attempt to construct a transit road network of intersections
and the roads between them, 
and map each route to a sequence of road segments.


The simplest way to construct such a network is to use the stop sequence:
routes with a common sub-sequence of stops must be traveling the same route between them,
so by defining a road as the path between two consecutive stops 
we can obtain travel time estimates along that road segment.
There are several places where this doesn't quite work, for example express routes 
and multi-stop locations where there are high numbers of routes,
resulting in some overlap between road segments,
it is a viable enough simplication for the purposes of this paper.
Figure~\ref{fig:network_creation} demonstrates how several overlapping routes 
are merged to form a road network.
In section~\ref{sec:kf} we present a model for estimating vehicle travel time
along a road segment.
Future work will look at using physical intersections as nodes in the network to improve it.

\begin{figure}[tb]
    \centering
    \begin{subfigure}{0.7\textwidth}
        \centering
        \includegraphics[width=0.95\textwidth]{figures/02_network_segments_1.pdf}
        \caption{Raw GTFS route shapes}
        \label{fig:network_creation_1}
    \end{subfigure} \\
    \begin{subfigure}{0.7\textwidth}
        \centering
        \includegraphics[width=0.95\textwidth]{figures/02_network_segments_2.pdf}
        \caption{GTFS-based road network}
        \label{fig:network_creation_2}
    \end{subfigure}
    \caption{Construction of a route network involves combining routes at nodes %
        (here we are using stops) which are connected by edges (roads). %
        In (a), the three unique routes, represented by different line types, clearly %
        overlap in several places. In (b) these have been merged, and the width of each line %
        represents how many routes use that link.}
    \label{fig:network_creation}
\end{figure}


\subsection{Realtime vehicle locations}
\label{sec:realtime_data}

GTFS \rt allows developers to query the current positions of vehicles
in the transit network.
The data consists of the time $t_k$ that the observation was made,
the GPS position of the vehicle, $\bY_k$, 
and some other information about the trip being serviced.
Vehicle positions are updated with a frequency of anywhere between 10~seconds and several minutes,
so there is often a lot of uncertainty about the trajectory
between two observations, particularly when there is one or more bus stop
or intersection between them.
It is also possible for a bus to remain stationary,
so the possible vehicle positions rapidly increases with the time between observations.


Another complication with the Auckland Transport realtime feed is that
the buses are programmed to report their location when arriving at
bus stops and some major intersections.
Often these updates are preemptive 
(i.e., the bus is almost there, but not quite),
and subsequent observations place the bus \emph{behind} the stop or intersection
(e.g., in a queue of traffic at traffic lights).
To handle this, we compute the approximate distance traveled, $\tilde x_k$,
of the vehicle by finding the nearest point on the path to the observation;
if this has decreased, the current state is rejected and the vehicle reverted
to its previous state.


\section{The Models}
\label{sec:models}

The real-time nature of this application (means that for) both
the vehicle and network models a recursive Bayesian filtering approach is logical,
and indeed real-time tracking and robotics applications often use such (models).
In each of the following models, 
we assume an underlying Markov process with state $\mathcal{X}_k$,
of which observations $\mathcal{Y}_k$ are made,
which leads us to the following general model,
\begin{equation}
\label{eq:rbe_model}
\begin{split}
\mathcal{X}_k &= f(\mathcal{X}_{k-1}, \omega_k) \\
\mathcal{Y}_k &= h(\mathcal{X}_k, \nu_k)
\end{split}
\end{equation}
where $\omega_k\sim\mathrm{N}(0,\mathcal{Q})$ is the system noise,
and $\nu_k\sim\mathrm{N}(0,\mathcal{R})$ is measurement error.
The transition function $f$ determines the relationship between consecutive states,
while the measurement function $h$ is a deterministic function describing the
relationship between the underlying and observed states.

The following sections describe two Bayesian filter models based on (\ref{eq:rbe_model}) used in this application.
The first model is implemented using a particle filter to estimate vehicle states,
with the primary objective of estimating vehicle travel times along roads,
while the second using these estimated times to update road states,
and is implemented using a Kalman filter.

\subsection{Real-time vehicle model}
\label{sec:pf}

The underlying vehicle state at time $t_k$ consists of
the vehicle's distance traveled $x$ in meters along the route and
its speed $\dot x$ in meters per second.
Additionally, we include travel times along road segments along the route, 
$\bz = (z_1,\ldots,z_L)^\top$, in seconds,
which are independent of time but included in the state as they are estimated
sequentially as the vehicle traverses the route.
These are combined into the state vector,
\begin{equation}
\label{eq:vehicle_state}
\bX_k = 
\begin{bmatrix}
    x_k \\ \dot x_k \\ \bz
\end{bmatrix}.
\end{equation}
Observations of the vehicle are made at time $t_k$ using a GPS,
giving the longitude $\lambda_k$ and latitude $\phi_k$ of the vehicle,
giving us the observation vector
\begin{equation}
\label{eq:vehicle_obs}
\bY_k = \begin{bmatrix} \lambda_k \\ \phi_k \end{bmatrix}.
\end{equation}


We decided to employ a particle filter to estimate $\bX_k$,
since it is a highly flexible approach and has been used in recent 
transit vehicle modeling applications \citep{Hans_2015}.
Our main justification for using it is that it handles multimodality very well,
which is a common feature of the proposal distribution particularly around bus stops.
Another advantage is the intuitive likelihood function, 
which is described later.
Conversely, the particle filter is a computationally demanding method,
as thousands of particles are required per bus.
Section~\ref{sec:rt} describes how we were able to implement the particle filter in real-time.


In the particle filter, the posterior distribution of the state at time $t_k$,
is estimated by a set of discrete points, or particles,
\begin{equation}
p(\bX_k | \bY_k) \approx \tilde\bX_{k|k} := (\bX_k^{(i)})_{i=1}^N
\end{equation}
each of which is independently updated or \emph{mutated} using the transition function $f$,
\begin{equation}
p(\bX_k | \bX_{k-1}) \approx \tilde\bX_{k|k-1} := 
\left(f(\bX_{k-1}^{(i)}, \psi)\right)_{i=1}^N
\end{equation}
using a parameter vector $\psi$ containing all of the necessary parameters
for the model (including system noise).
After mutation, \emph{selection} of a new set of particles is performed by
importance resampling using likelihood weights.


\subsubsection{Vehicle transition function}
\label{sec:pf_prediction}
As we are using a particle filter to implement the vehicle model,
the transition function $f$ can be a complex description of bus behaviour,
most notably
\begin{itemize}
\item non-constant speed along roads (acceleration as system noise),
\item stopping and waiting at bus stops while passengers board and disembark, and
\item stopping and waiting at intersections.
\end{itemize}
The details of this are given in the algorithm defined in the Appendix.

For each particle, the transition function generates a plausible trajectory,
using system noise parameter $\sigma^2$ which describes 
how the vehicles acceleration changes as a random process,
and propogated using Newton's laws of motion \cite{}
\begin{equation}

\end{equation}
This includes stopping probabilities $\boldsymbol\pi = (\pi_1,\ldots,\pi_J)^\top$ at bus stops,
dwell times $\boldsymbol\tau = (\tau_1,\ldots,\tau_J)^\top$ for passengers to
board and disembark (conditional on the bus stopping),
and the minimum dwell time at stops, $\gamma$.
For these parameters, we used constant values for all stops,
and based the values on those used by \cite{Hans_2015};
future work will look at modeling these separately in real-time also.

Also in this application, due to the complexity of the problem,
we use bus stops to define road segments.
Therefore no intersection model is required,
only the segment travel times as predicted by the network model at time $t_k$,
$\btheta(t_k) = (\theta_1(t_k), \ldots, \theta_L(t_k))^\top$
(see section~\ref{sec:kf}).
Future work will adapt this to use intersection that are indpendent 
of bus stops, and therefore completely independent of routes,
which will allow increasing the complexity of the model to incorporate
intersection stopping probabilities and waiting times.


% The vehicle model involves infering the \emph{unobservable state} $\bX_k$ 
% of a vehicle at time $t_k$ from the \emph{observed state} $\by_k$,
% which in this case is the GPS location of the vehicle,
% where $\lambda_k$ and $\phi_k$ are the longitude and latitude, respectively.
% \begin{equation}
% \label{eq:vehicle_observation}
% \by_k = \begin{bmatrix} \lambda_k \\ \phi_k \end{bmatrix}
% \end{equation}
% To construct the unobservable state, 
% each vehicle is assumed to follow the trajectory defined by a trajectory function of time, $x(t)$,
% such that the distance traveled along a route by a vehicle at time $t_k$ is given by
% $x_k = x(t_k)$.
% Additionally, speed and acceleration are given by
% the first and second derivatives of the trajectory function,
% $\dot x_k = x'(t_k)$ and $\ddot x_k = x''(t_k)$, respectively.
% Exactly how the state is constructed will depend on the model being used.


% Since observations of the vehicle are irregular, 
% a \emph{transition function} $f$ is used to
% predict the vehicle's state at time $t_k$ given the current knowledge of the state at time $t_{k-1}$,
% as well as accounting for \emph{system noise}, $\sigma_x^2$.
% We also define a \emph{measurement function} $h$ which describes the relationship between
% the observable and unobservable states,
% and accounting for \emph{measurement error}, $\bv_k$,
% \begin{equation}
% \label{eq:vehicle_measurement}
% \by_k = h(\bX_k) + \bv_k = h(x_k) + \bv_k,
% \end{equation}
% so all models need to estimate at least $x_k$, the distance traveled along the route.


% The model is split into two steps, prediction and update.
% The prediction involves using the transition function to predict
% the next state, $\hat\bX_{k|k-1}$,
% while the update step using a likelihood function and the measurement function
% to update the state to account for the observed vehicle location, $\by_k$,
% giving a final estimate of $\hat\bX_{k|k}$.


% \subsubsection{Predicting state using the transtion function}
% \label{sec:pf_prediction}

% The transition function is a model of bus behaviour that allows
% predictions of future states to be made given the most recently 
% updated state.
% Developing a good model for bus behaviour allows us to estimate
% parameters of interest, such as vehicle speed,
% dwell time at bus stops, and travel time along individual roads,
% the last of which is of course our primary objective.
% In this section, we propose [HOW MANY?] increasingly complex models
% of bus behaviour.


% %% MODEL 1
% The first model, $\bM_0$, assumes a constant speed between observations
% using the state
% $\bX_k = \left[\begin{smallmatrix}x_k && \dot x_k\end{smallmatrix}\right]^\top$
% and Gaussian system noise with mean zero and variance adjusted by
% $\Delta_k = t_k - t_{k-1}$.
% \begin{equation*}
% \hat\bX_{k|k-1} = f_0(\bX_{k-1|k-1}, \Delta_k, \sigma_x^2) =
% \begin{bmatrix}
% x_{k|k-1} & \dot x_{k|k-1}
% \end{bmatrix}
% \end{equation*}
% where
% \begin{align*}
% x_{k|k-1} &= x_{k-1|k-1} + \Delta_k \dot x_{k|k-1} \\
% x_{k|k-1} &= \dot x_{k-1|k-1} + w_k \\
% w_k &\sim \mathcal{N}_T(0, \Delta_k \sigma_x^2)
% \end{align*}
% Here, $\mathcal{N}_T$ denotes a truncated normal distribution such that
% the system noise is truncated to ensure the speed remains between 0 and 30~m/s.


% While $\bM_0$ may perform well on average,
% and indeed if all we were interested in was estimating $x_k$ it may be adequate;
% however, it is missing two important features of bus behaviour:
% \begin{itemize}
% \item variable speeds along roads, \emph{which is what we are truly interested in}
% \item unknown dwell times at bus stops
% \end{itemize}


% The first of these is incorporated into $\bM_1$,
% which allows vehicle speeds to vary over time.
% This model uses the same state as $\bM_0$,
% but a different transition function $f_1$.
% In this case, the state equations depend on a sequence of system noise,
% $\bw_k = \{\bw_k^1, \cdots, \bw_k^{\Delta_k}\}$
% \begin{align}
% \label{eq:transition_f1}
% x_{k|k-1} &= x_{k-1|k-1} + \Delta_k \dot x_{k-1|k-1} + \sum_{i=1}^{\Delta_k} w_k^i \nonumber \\
% \dot x_{k|k-1} &= \dot x_{k-1|k-1} + \sum_{i=1}^{\Delta_k} w_k^i \\
% w_k^i &\sim \mathcal{N}_T(0, \sigma_x^2) \nonumber
% \end{align}
% Again, system noise is assumed to have a Gaussian distribution, 
% truncated to ensure that the vehicle's speed is always between 0 and 30~m/s.
% This is achieved by iteratively estimating each second the the vehicle's state,
% each time ensuring the speed remains in the specified range.


% The next model, $\bM_2$, allows for dwell times at bus stops.
% Since buses can either stop and wait, or not stop,
% multimodality is introduced to the system,
% which requires a modeling framework, such as the particle filter used here,
% to handle it.
% For each bus stop $j = 1, \ldots, M$ along a route,
% its distance into the trip, $S_j^d$, 
% is used to determine when a vehicle passes a stop.
% When this occurs, the dwell time $\delta_j \geq 0$ at stop $j$ is modelled
% using the stopping probability, $\pi_j$,
% a common dwell time parameter $\gamma$ which models the
% deceleration, opening and closing of doors, and acceleration at bus stops,
% and a stop dwell time $\tau_j$ for passengers alighting and disembarking.
% Therefore the stop dwell time is
% \begin{equation}
% \label{eq:dwell_time}
% \delta_j = p_j(\gamma + \tilde\delta_j),\quad
% p_j \sim \mathrm{Bern}(\pi_j),\quad
% \tilde\delta_j \sim \mathcal{E}(\tau_j)
% \end{equation}

% To define the transition function for this model, $f_2$,
% we need to account for multiple dwell times.
% Let $s_n$ be the index of the next stop,
% and sample dwell times for each upcoming stop $j = s_n, \ldots, M$,
% noting that these can be zero.
% Assuming the vehicle passes stops $s_n$ up to $J-1$,
% does it reach stop $J$ in the time remaining?
% \begin{equation}
% \tilde X_k^J = \min\left\{
% S_J^d, f_1(\hat X_{k|k-1}, \Delta_k - \sum_{j=s_n}^{J-1}\delta_j)\right\}
% \end{equation}
% Taking the maximum of the sequence gives the predicted distance
% traveled by the vehicle, so
% \begin{equation}
% \hat X_{k|k-1} = f_2 (\hat X_{k-1|k-1}, \Delta_k) =
% \max_{J = s_n, \ldots, M}\left\{
%     \tilde X_k^J
% \right\}
% \end{equation}

% Care needs to be made when generating the sequence,
% so in practice the state will be calculated recursively to ensure
% the system noise is consistent.
% \begin{equation*}
% \hat{\tilde X}_k^J =
% \begin{cases}
%     f_1(\hat{\tilde X}_k^{J+1}, \delta_J) & J < M \\
%     \hat X_{k-1|k-1} & J = M
% \end{cases}
% \end{equation*}

% The final model $\bM_3$ demonstrated here will allow acceleration to vary.
% \begin{align*}
% \label{eq:model_m3}
% x_{k|k-1} &= x(t_k) \\
% \intertext{where states are recursively calculated back to $t_{k-1}$}
% x(t) &= x(t - 1) + \dot x(t) \\
% \dot x(t) &= \dot x(t-1) + \ddot x(t) \\
% \ddot x(t) &= \ddot x(t-1) + w_t,\quad
% w_t \sim \mathcal{N}(0,\sigma_x^2)
% \end{align*}
% By truncating the acceleration, 
% the speed of the vehicle can be maintained between 0 and 30~m/s.
% Additionally, when approaching a bus stop that will be stopped at,
% the acceleration will be trunctated to below zero,
% however no physical deceleration will be modeled as this is too complex
% and stopping behaviour of buses can be sudden 
% (for example when passengers signal late that they wish to board or disembark).
% After servicing a stop, the bus will need to accelerate up to speed again. 


% \subsubsection{Updating state using the observation likelihood}
% \label{sec:pf_update}

% The update step involves comparing the predicted state $\bX_{k|k-1}$
% with the observed state $\bY_k$,
% using the measurement function $h$ to convert between the two state spaces,
% and a likelihood function.
% The measurement function is a deterministic function that uses only the 
% distance traveled, $x_k$, along a known path $\bR$ to obtain a GPS coordinate.
% The inverse measurement function, which is used in other approaches,
% is \emph{not deterministic}, and instead uses an optimisation to
% find the ``estimated'' distance traveled,
% which invariably leads to issues in the implementation.
% By using a particle filter approach, we avoid using $h^{-1}$
% and instead determine the liklihood using $\bY_k$ directly.

% For the likelihood, the GPS error is assumed symmetrical about the true
% position of the vehicle, with a radius of $\sigma_y^2$ meters.
% Define a projection function $g(\cdot | \bY)$ that projects points
% onto a cartesian surface (``Flat Earth''),
% so that the distance between points is the real Earth distance.
% Then the distance between the points is
% \begin{align*}
% \delta_k &= \sqrt{\left(g(h(X_k))_1 - g(\bY_k)_1\right)^2 +
%     \left(g(h(X_k))_2 - g(\bY_k)_2\right)^2} \\
% \intertext{which can be standardised by the GPS error}
% \tilde \delta_k &=
% \sqrt{\left(\frac{g(h(X_k))_1 - g(\bY_k)_1}{\sigma_y}\right)^2 +
%     \left(\frac{g(h(X_k))_2 - g(\bY_k)_2}{\sigma_y}\right)^2} =\frac{\delta_k}{\sigma_y} \\
% \intertext{Since we assume symmetric variance, this is actually}
% \tilde\delta_k &= \sqrt{Z_1^2 + Z_2^2}, \quad
% Z_1,Z_2 \sim \mathcal{N}(0,1)
% \end{align*}
% The sum of two normal random variables has an Exponential distribution with rate $\frac{1}{2}$,
% therefore 
% \begin{equation}
% \label{eq:lhood_exp}
% \left(\frac{\delta_k}{\sigma_y}\right)^2 \sim \mathrm{Exp}\left(\frac{1}{2}\right)
% \end{equation}

% Combining this together, we use a distance function $d(\bY_1, \bY_2)$ that calculates
% the geographic distance between two GPS coordinates,
% the likelihood is
% \begin{equation}
% \label{eq:lhood}
% f(\bY_k | \hat\bX_{k|k-1}, \sigma_y^2) =
% \frac{1}{2} \exp\left\{-\frac{d(\bY_k, h(\hat\bX_{k|k-1}))^2}{2\sigma^2}\right\}
% \end{equation}


\subsection{Network model}
\label{sec:kf}

% The second part of this appraoch involves a network model of transit network travel times along links in the network.
% Reliable ETAs will require a historical data component,
% used as a prior in the absense of data as well as for
% predicting future states, 
% and a realtime update component where the travel times of vehicles
% are used to update the current estimate of network state.

% The model will be another recursive Bayesian model as used for the vehicle state,
% as this provides a simple way to model realtime data.
% Since travel times can safely be assumed to be unimodal,
% we use an information filter (IF),
% which is a transformation of the Kalman filter
% using information instead of covariance,
% and allows the simple combination of multiple observations,
% for example when multiple buses traverse a road in the same update.
% The prediction step will involve a transition function that incorporates historical data
% and the update step will weight between observed and predicted travel times.

% The network state $\Theta_c = \{\theta_c^j\}_{j = 1}^J$ is the travel time 
% of transit vehicles along road segment $j$ at time $t_c$.
% In this implementation, we are assuming independence between each segment
% to provide a simple basis for the model; of course in reality this
% is not true and future work will investigate ways of incorporating this.

% Using an IF requires a transition matrix, $\mathbf{F}_c$;
% however, in this case the travel time model is constant so no
% transition matrix is required (i.e., $\mathbf{F}_c = 1$).
% Let $\Delta_c = t_c - t_{c-1}$ and $P_c^j$ the state uncertainty
% with system noise $Q_c = \Delta_c \sigma_b^2$, then the update equations are
% simply the normal KF update equations,
% \begin{align}
% \label{eq:kf_transition}
% \hat\theta^j_{c|c-1} &= \hat\theta^j_{c|c-1} \\
% P^j_{c|c-1} &= P^j_{c-1|c-1} + Q_c
% \end{align}

% For the update step, however, we need to transform into an information
% space parameterised by the information matrix 
% $\mathbf{Z}^j_{c|c-1} = P_{c|c-1}^{-1}$ 
% and the information vector $\mathbf{\hat z}^j_{c|c-1} = \hat\theta_{c|c-1} P_{c|c-1}^{-1}$.
% The measurement data obtained from the vehicle model are
% the travel time of vehicle $m$ along segment $j$,
% $\bar b^m_j$, and the uncertainty $s^m_j$.
% These can be transformed to a measurement information covaraince matrix
% and vector using the measurement matrix $\mathbf{H} = 1$ since,
% in this case, the observed state is the underlying state we are estimating (travel time).
% \begin{equation}
% I^m_{jc} = \frac{1}{(s^m_j)^{2}}\quad\text{and}\quad
% i^m_{jc} = \frac{\bar b^m_j}{(s^m_j)^2}
% \end{equation}
% The information update is now the sum of the information for all vehicles
% that traversed the segment since the last update, so
% \begin{align*}
% \mathbf{Z}^j_{c|c} &= \mathbf{Z}^j_{c|c-1} + \sum_{m=1}^M I^m_{jc} \\
% \mathbf{\hat i}^j_{c|c} &= \mathbf{\hat z}^j_{c|c-1} + \sum_{m=1}^M i^m_{jc}
% \end{align*}
% which are easily transformed back to get the desired state parameters.
% \begin{equation}
% \hat\theta^j_{c|c} = \frac{\mathbf{\hat z}^j_{c|c}}{\mathbf{Z}^j_{c|c}} 
% \quad\text{and}\quad
% P^j_{c|c} = \frac{1}{\mathbf{Z}^j_{c|c}}
% \end{equation}




\section{Real-time implementation and preliminary results}
\label{sec:rt}

There are two components to the application:
the static GTFS~schedule and network data,
and the real-time modeling and prediction.
We chose \verb+Rcpp+ to develop our program,
giving us the advantages of an R~package for data manipulation 
(notably \verb+RSQLite+ and \verb+dplyr+)
and distribution, 
as well as the speed and memory management capabilities of \verb|C++|. 
As a result, all of the following has been implemented in the R package
\verb+transitr+, available on Github (\url{https://github.com/tmelliott/transitr}).
Here our focus is on the real-time aplication, so we are only discussing the real-time 
features of model's implementation.

The general structure of the main function is:
\begin{enumerate}
\item Load GTFS data from database
\item Indefinitely repeat when new data recieved \ldots
\begin{enumerate}
    \item Update or create new vehicle objects from new data
    \item Run particle filter on each vehicle to estimate update state
    \item Collect travel time information for each vehicle for any completed roads
    \item Update road network with any completed road segments
    \item Generate ETAs for vehicles
    \item Write ETAs to extended Google Protobuf binary file for distribution
\end{enumerate}
\end{enumerate}

The main concern is speed: 
we desire ETAs to be available as soon as possible after obtaining the data.
To do this, we use raw pointers to read-only GTFS objects,
and particles are only copied when resampling is required.
This is also why we chose to use protobuf as the output format,
since this is fast to write and distribute.
Steps 2b--e are performed in parallel, 
which in our test environment of a virtual machine with 8~cores, 
significantly speeds up computation.


As far as results, we have two categories:
implementation and faesibility,
and model performance;
we do not have any complex arrival time estimation method,
so no results for that are shown here.


% How we implement it, choice of software (Rcpp = R + C++).
% R: dealing with data structures is easier, maintainability, interfacing
% C++: speed

% Overall structure:
% - load
% - fetch positions
% - initialize or mutate+update
% - update network
% - make ETA predictions (vehicle state + network state)

% Some of the key things:
% - minimise copy, parallelisation using OMP
% - moving as much computation ``outside'' of the main loop as possible
%   (e.g., ``pre''-predict vehicle/network states so only update required)
%   so ETAs are generated ASAP after retrieving data
% - keeping the GTFS database up-to-date by fetching new data each morning
% - distribution - a cloud database vs a single protobuf file with everything 
%   (maintenance/reliability/speed/size)


% \section{Preliminary Results}
\label{sec:results}

Currently only have timings (i.e., yes this is plausible).
- Function of number of particles/number of cores/number of buses.
- Network coverage? i.e., how many road segments actually get enough data
to generate useful numbers

- any problems?


\section{Discussion and Future Work}
\label{sec:discussion}

The main goal of this work was to explore the faesibility
of a particle filter as a \rt estimation method for transit vehicle states,
and the use of estimated states to determine traffic conditions
throughout the network.
This allows vehicles to inform others---irrespective of route---of 
how long it will take to reach future stops.


Not only is the particle filter a viable choice in terms of speed,
it has shown it can handle most situations without losing the vehicle,
with the exception of when the vehicle is not traveling along
the specified route.
Hopefully with future development,
the stopping behaviour can be more accurately modeled,
perhaps incorporating trip updates into the likelihood function.


The network model is still in the early stages of development,
but given that the implementation is running quickly---twice as fast as our target---%
there is room for developing a more complex model.
Future work will explore the use of historical data to develop informative priors
for travel times,
which will provide a useful fallback in the case of no data,
or for making more accurate long-term predictions.
Once such a model has been developed, we plan to explore prediction estimates,
particuarly comparing point and intervals as a means of communicating 
uncertainty to commuters.

\bibliographystyle{anzsj}
\bibliography{../reflist.bib}

\end{document}
