\documentclass[times, doublespace]{anzsauth}
\usepackage{moreverb}
\usepackage{url}
\usepackage{grffile}
\usepackage{lineno}
% \usepackage{lmodern}
\usepackage{lipsum}
\usepackage[UKenglish]{isodate}

\usepackage[acronym]{glossaries}
\newacronym[shortplural={ATIS}]{atis}{ATIS}{advanced traveller information system}
\newacronym{avl}{AVL}{automatic vehicle location}
\newacronym{apc}{APC}{automatic passenger counter}
\newacronym{rti}{RTI}{real-time information}
\newacronym{gps}{GPS}{Global Positioning System}
\newacronym{api}{API}{application programming interface}
\newacronym{gtfs}{GTFS}{general transit feed specification}
\newacronym{knn}{KNN}{$k$-nearest neighbour}
\newacronym{ann}{ANN}{artificial neural networks}
\newacronym{svm}{SVM}{support vector machines}
\newacronym{kf}{KF}{Kalman filter}
\newacronym{ekf}{EKF}{extended Kalman filter}
\newacronym{pf}{PF}{particle filter}
\newacronym{mcmc}{MCMC}{Markov chain Monte Carlo}
\newacronym[longplural={expected times of arrival}]{eta}{ETA}{expected time of arrival}
\newacronym{at}{AT}{Auckland Transport}
\newacronym{pt}{PT}{public transport}
\newacronym{ui}{UI}{user interface}
\newacronym{osm}{OSM}{OpenStreetMap}

\def\volumeyear{2018}
\linenumbers

\begin{document}
\cleanlookdateon
\runningheads{Short title of paper}{TOM~ELLIOTT AND THOMAS~LUMLEY}
\title{Long name of the paper}
\author{Tom Elliott\corrauth and Thomas Lumley}
\affiliation{University of Auckland}
\address{
    Department of Statistics, University of Auckland, Private Bag 92019, Auckland 1142, New Zealand\\
    Email: \texttt{tom.elliott@auckland.ac.nz}
}

\begin{abstract}
Lorem ipsum blah blah blah.
\end{abstract}

\keywords{particle filter; Kalman filter; transit modeling;
          transit networks; statistical computing; applications}

\maketitle
\section{Introduction}
\label{sec:intro}

Bus arrival-time prediction is far from a novel concept: 
ever since the first vehicle tracking technology was implemented
in transit vehicles in the late 1980's, 
real-time models have been implemented to provide \gls{rti} for commuters.
The simplest form of this is, the \gls{eta}
(usually displayed as \emph{minutes until arrival}),
was implemented in {{region}} by \cite{Wall99analgorithm} 
using their proposed \gls{kf} algorithm.
In their work, they used historical data to obtain \glspl{eta}.

Over the years, advances in computing power have enabled new
modeling approaches,
such as support vector machines \citep{Yu_2006},
artificial neural networks \citep{Yu_2011},
and more recently particle filters \citep{Hans_2015}.



\section{GTFS network construction}
\label{sec:gtfs}

Before we can usefully model buses in real-tim{}e, 
we need to construct a \emph{transit network},
consisting of \emph{intersections} (nodes)
connected by \emph{road segments} (edges).
In this way, each \emph{route} 
(a journey taken by a transit vehicle, from an origin to a destination by a fixed path)
can be represented as a sequence of road segments,
each of which we model as vehicles travel along them (see section~\ref{sec:kf}).




\section{Real-time vehicle model}
\label{sec:pf}

The real-time vehicle model that runs ... in real time?


\section{Network model}
\label{sec:kf}

Two parts to this - first, the prior prediction step; then, the data update step.

\subsection{Prediction step}
\label{sec:kf-pred}



\subsection{Update step}
\label{sec:kf-update}




\bibliographystyle{anzsj}
\bibliography{../reflist.bib}

\end{document}
