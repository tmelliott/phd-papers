\documentclass[times, doublespace]{anzsauth}
\usepackage{moreverb}
\usepackage{url}
\usepackage{grffile}
\usepackage{lineno}
\usepackage{lipsum}
\usepackage[UKenglish]{isodate}

\def\volumeyear{2018}
\linenumbers

\begin{document}
\cleanlookdateon
\runningheads{Short title of paper}{TOM~ELLIOTT AND THOMAS~LUMLEY}
\title{Long name of the paper}
\author{Tom Elliott\corrauth and Thomas Lumley}
\affiliation{University of Auckland}
\address{
    Department of Statistics, University of Auckland, Private Bag 92019, Auckland 1142, New Zealand\\
    Email: \texttt{tom.elliott@auckland.ac.nz}
}

\begin{abstract}
Lorem ipsum blah blah blah.
\end{abstract}

\keywords{particle filter; Kalman filter; transit modeling;
          transit networks; statistical computing}

\maketitle
\section{Introduction}
\label{sec:intro}

Overview of the history, problem, proposed solution I guess...


\section{GTFS network construction}
\label{sec:gtfs}

Before we can usefully model buses in real-time, 
we need to construct a \emph{transit network}, 
consisting of \emph{intersections} (nodes),
connected by \emph{road segments} (edges).
In this way, each \emph{route} 
(a journey taken by a transit vehicle, from an origin to a destination by a fixed path)
can be represented as a sequence of road segments,
each of which we will model as vehicles travel along them (see section~\ref{sec:kf}).


\section{Real-time vehicle model}
\label{sec:pf}

The real-time vehicle model that runs ... in real time?


\section{Network model}
\label{sec:kf}

Two parts to this - first, the prior prediction step; then, the data update step.

\subsection{Prediction step}
\label{sec:kf-pred}



\subsection{Update step}
\label{sec:kf-update}





\end{document}